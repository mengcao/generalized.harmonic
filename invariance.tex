\documentclass[letterpaper,nofootinbib,prd,amsmath,onecolumn]{revtex4-1}

\usepackage{amsmath}
\usepackage{amsthm}
\usepackage{graphicx}
\usepackage{float}
%\usepackage{listings}
%\usepackage{pdfpages}
\usepackage{courier}
\usepackage{color}
\usepackage{mathrsfs}

\begin{document}
\special{papersize=8.5in,11in}

\title{Covarian Gauge Conditions for Foliation Preserving Coordinate Transformations}
\author{J.~David Brown, Meng Cao}
\affiliation{Department of Physics, North Carolina State University, Raleigh, NC 27695 USA}

\begin{abstract}
The transform rules of 3 + 1 splitting variables, such as the lapse, shift vector, extrinsic curvature and etc., under a foliation preserving coordinate transformation are presented. The covariant forms of time derivative of the lapse and time derivative of the shift vector are given. The invariant form of 1 + log slicing condition and Gamma-driver shift equation are derived from those covariant time derivatives. Utilize these invariant gauge conditions, it allows one to evolve the same system with two different coordinate systems and reach the same result. A time dependent spatial rotation coordinate transformation is discussed in details as a specific example of this case.  
\end{abstract}
\maketitle

%%%%%%%%%%%%%%%%%%%%%%%%%%%%%%%%%%%%%%%%
\section{Introduction}
The Einstein's equations are in the covariant form under spacetime diffeomorphism, which means the equations are invariant under spacetime coordinate transformations. 3 + 1 splitting mechanism slices the spacetime into a family of spacelike hypersurfaces which is referred as foliations. On each foliation, a timelike parameter t is constant. By breaking the symmetry of time and space domain, the 3 + 1 formulation makes it possible to evolve a general relativity system along with time numerically. On the other hand, although it still preserves the invariant property, it is not manifest in the equations anymore. While constructing numerical simulations, either in ADM formulation or BSSN formulation, people use gauge conditions(such as 1 + log slicing and Gamma-driver shift condition) to evolve the lapse and shift vector along with time. Unfortunately, these gauge condition equations are neither explicitly nor implicitly covariant. This scheme works sufficiently accurate if the system is evolved in one single coordinate system. However, if the same system is evolved with two different coordinate systems to the same time under this kind of gauge conditions, it is not guaranteed that we can get the same results. 

This situation leads to our motivation of this paper. We will try to examine the transform pattern of physical quantities in 3 + 1 splitting system under a so called ``foliation preserving coordinate transformation'', which includes time reparamerization and time dependent spatial deffeomorphism while the structure of foliations is preserved. Then we will try to write down the covariant form of the gauge conditions under this coordinate transformation and give the guideline of how to implement this into numerical simulation. 

The paper is organized as follows. In Sec.~\ref{3+1}, we explain the notation and convention we are going to follow in this paper. Then we briefly review the 3 + 1 splitting formulation while introducing several significant terms which will be used in latter sections. In Sec.~\ref{transform}, we give the definition of foliation preserving coordinate transformation, under which we derive and list the transform pattern of all significant 3 + 1 splitting physical quantities. The ordinary gauge conditions, i.e.~1 + log slicing condition and Gamma-driver shift equations, are dicussed in details in Sec.~\ref{gauge}. They are then modified to covariant forms and a guideline of how to implement this covariant gauge condition is provided. Last but not least, in Sec.~\ref{example} we demonstrate how this scheme works in a specific coordinate transformation case, time dependent spatial rotation.
%%%%%%%%%%%%%%%%%%%%%%%%%%%%%%%%%%%%%%%%%%%%%%%%%%%%%%%%%%%%%%%%%%
\section{General covariance}
Einstein's theory is generally covariant, That is, Einstein's field equations take the same 
form in any spacetime coordinate system (reference Weinberg, Wald? others?). More precisely, the Einstein 
equations written in the form $G_{\mu\nu} = 8\pi G T_{\mu\nu}$ are {\em manifestly covariant}. What we mean 
by manifest covariance is that the transformation properties of each term in the equations makes it obvious that the 
form of the equations is unchanged when we change coordinate systems. For example, each of the terms $G_{\mu\nu}$ and 
$T_{\mu\nu}$ transforms as a type $0\choose 2$ tensor under changes of spacetime coordinates. Thus, under the coordinate 
transformation $x^{\mu'} = x^{\mu'}(x^\mu)$, the Einstein tensor becomes
\begin{equation}
	G_{\mu\nu} = \frac{\partial x^{\mu'}}{\partial x^{\mu}} \frac{\partial x^{\nu'}}{\partial x^{\nu}} G_{\mu'\nu'} \ .
\end{equation}
A corresponding expression holds for the stress--energy tensor $T_{\mu\nu}$. By applying the transformation 
to all terms in the Einstein equations we find that the factors 
$(\partial x^{\mu'}/\partial x^{\mu}) (\partial x^{\nu'}/\partial x^{\nu})$
cancel, leaving $G_{\mu'\nu'} = 8\pi G T_{\mu'\nu'}$. 

Now consider the $3+1$ splitting, with matter fields set to zero for simplicity. We have the ADM (York) equations 
\begin{subequations}
\begin{eqnarray}
	\partial_t g_{ab} - {\cal L}_\beta g_{ab} & = & -2\alpha K_{ab} \ , \\
	\partial_t K_{ab} - {\cal L}_\beta K_{ab} & = & \alpha K K_{ab} - 2\alpha K_{ac}K^c_b 
		+ \alpha R_{ab} - D_a D_b\alpha \ ,
\end{eqnarray}
\end{subequations}
where $K_{ab}$, $g_{ab}$, $\alpha$, and $\beta^a$ are the extrinsic curvature, spatial metric, lapse and shift, respectively, 
and $D_a$ is the spatial covariant derivative. 
These equations hold in any spacetime coordinate system. However, they are not 
manifestly covariant because they are written in terms of new 
fields, $K_{ab}$, $g_{ab}$, $\alpha$ and $\beta^a$, that do not transform as tensors under 
spacetime coordinate transformations. The ADM equations are generally covariant but 
not manifestly so. 

Although the ADM equations are not manifestly covariant under {\em spacetime} coordinate transformations, they are 
manifestly covariant under changes of {\em spatial} coordinates. This is seen from the fact that each term 
transforms as a type $0\choose 2$ tensor under spatial coordinate transformations. In fact, we will show that these 
equations are manifestly 
covariant under {\em time--dependent} spatial coordinate tranformations as well as time reparametrizations. Together, these 
transformations form the subset of spacetime coordinate transformations that preserve the foliation of spacetime 
into spacelike hypersurfaces. Thus, we refer to these transformations as ``foliation preserving". 

The 3+1 split equations is the starting point for the initial value problem and the Hamiltonian formulation. And 
of course numerical treatments. The lack of manifest covariance is not a problem. The point of view has shifted. 
We now view the theory as a set of fields on 3d space, that evolve in time. Thus the emphasis on invariance is 
in space. But just as the spatial fields evolve in time, so can the coordinate system. And the time labels 
are not physical, so we should be able to relabel these. Thus we focus on the 
``foliation preserving" transformations. 

Say words about what invariance means in the 3+1 context. Really, a coordinate system has little meaning apart from 
topological restrictions. For example, if our manifold is Real 4, or $S^4$ in which case we would need periodic identifications. 
What does have definite meaning is the comparison of two different coordinate systems. Say $x^{\mu}$ and $x^{\mu'}$ are 
related by $x^{\mu'} = x^{\mu'}(x^\mu)$. We want to evolve in one coordinate system, then the other. We want the results of the 
two simulations to be related by the given transformation. We do this by using the transformation to relate the data at the 
intial time. Let's assume that the initial time is $t = t' = 0$, so the initial slices coincide. Then we need to use the 
transformation to show how the lapse and shift are related. In the end we should be able to 
relate the results according to the transformation. We also assume at this stage that the foliations coincide? If not, the comparison 
is very difficult; we would need to construct the 4D metric. Practically speaking, we restrict ourselves to the same slices, 
we again are focused on foliation preserving transformations. Then we compare spatial metric and extrinsic curvature. 

Display the coordinate transformation: 
\begin{align*}
t' & = t'(t)\\
x^{a'} & = x^{a'}(t,x^{a})
\end{align*}
These are foliation preserving. 

Now with BSSN, we introduce object as independent variables that are not spatial tensors $\tilde\Gamma^a$. 
And $\tilde g$ is a scalar, which means $\tilde g_{ab}$ is a type $0\choose 2$ tensor density of 
weight $-2/3$. The equations can be applied in any coordinate 
system, but it is difficult to  compare results unless we re-combine into simple tensor objects. 
We thus loose manifest covariance under foliation preserving transformations. All this assumes that we choose the 
lapse and shift in the appropriate way, as dictated by the spacetime coordinate transformation. 

Much the same for GH. In spacetime form, introduce 
a nontensor; in 3+1 form we have non spatial tensor $\Gamma^a$. Part of the purpose of this paper is to rewrite 
these equations in such a way that manifest covariance is restored. (Done in my other papers, right?) 

Note we are not dealing with gauge conditions yet. Gauge conditions break invariance. In particular, the 
1+log slicing keeps time--independent spatial coordinate invariance, but does not preserve time 
reparametrizations (or time--dependent spatial?). The gamma driver shift breaks everything. Then we 
can no longer compare results from two different coordinate systems. 

Our goal is to restore manifest covariance to the gauge and non--guage equations, under 
foliation preserving transformations. 


%%%%%%%%%%%%%%%%%%%%%%%%%%%%%%%%%%%%%%%%%%
\section{3 + 1 Splitting}\label{3+1}
Throughout this paper, we use Greek letter indices $\mu$, $\nu$, ... to denote spacetime indices and Latin letters $a$, $b$, ... to denote spatial indices. $\nabla_{\mu}$ stands for the spacetime covariant derivative while $D_{a}$ is the spatial covariant derivative.

In 3 + 1 formulation, we have
\begin{equation*}
	{}^{(4)}g_{\mu\nu} = (-\alpha^2 + \beta^{c}\beta_{c})\delta_\mu^0\delta_\nu^0 
	+ 2\beta_{b}\delta_{(\mu}^a\delta_{\nu)}^0 + g_{ab} \delta^a_\mu \delta^b_\nu \ ,
\end{equation*}
where $\alpha$ is called the lapse function, $\beta^{a}$ the shift vector and $g_{ab}$ the spatial metric. The covariant normal to each foliation is denoted as
\begin{equation}\label{normal covector}
n_{\mu} = -\alpha\delta^{0}_{\mu}
\end{equation}
 And its dual is
\begin{equation}\label{normal vector}
n^{\mu} = (\delta^{\mu}_{0} - \beta^{c}\delta^{\mu}_{c})/\alpha
\end{equation}
so that we have $n_{\mu}n^{\mu} = 1$. 

There is also the projection operator
\begin{equation}\label{projection 1}
X^{\mu}_{a} = \delta^{\mu}_{a}
\end{equation}
that projects a spacetime covector into a spacelike covector. Its covariant form is
\begin{equation}\label{projection 2}
X^{a}_{\mu} = \delta^{a}_{\mu} + \beta^{a}\delta^{0}_{\mu}
\end{equation}
so that we have $X^{\mu}_{a}X^{b}_{\mu} = \delta^{b}_{a}$ and $X^{\mu}_{a}n_{\mu} = 0$. 

With these definitions, the spacetime metric can be written in terms of the spatial metric, the normal and the projection operator as
\begin{equation}\label{spacetime metric 3 + 1}
^{(4)}g_{\mu\nu} = g_{ab}X^{a}_{\mu}X^{b}_{\nu} - n_{\mu}n_{\nu}
\end{equation}
Spacetime indices $\mu$, $\nu$, ... are always raised and lowered with the spacetime metric $^{(4)}g_{\mu\nu}$ and its inverse $^{(4)}g^{\mu\nu}$, while spatial indices $a$, $b$, ... are always raised and lowered with the spatial metric $g_{ab}$ and its inverse $g^{ab}$. 

Another important quantity in 3 + 1 splitting, the extrinsic curvature is defined as
\begin{equation}\label{extrinsic}
K_{ab} = - \frac{1}{2\alpha}\left[\partial_{t}g_{ab} - 2D_{(a}\beta_{b)}\right]
\end{equation}

%%%%%%%%%%%%%%%%%%%%%%%%%%%%%%%%%%%%%%%%%%%%%%%%%%%%%%%
\section{Foliation Preserving Coordinate Transformation}\label{transform}
This foliation preserving coordinate transformation is a time dependent spatial diffeomorphism combined with time reparameterization. We use $x^{\mu}$ to denote the original coordinate basis and $x^{\mu'}$ the transformed coordinate basis. Therefore, the transformation can be described as
\begin{align*}
t' & = t'(t)\\
x^{a'} & = x^{a'}(t,x^{a})
\end{align*}
and vice versa, we have
\begin{align*}
t & = t(t')\\
x^{a} & = x^{a}(t',x^{a'})
\end{align*}
Use the chain rule of partial differentiation to inspect the delta function for this tranformation, we have
\[
\delta^{a}_{b} = \frac{\partial x^{a}}{\partial x^{b}} = \frac{\partial x^{a}}{\partial x^{\mu'}}\frac{\partial x^{\mu'}}{\partial x^{b}} = \frac{\partial x^{a}}{\partial t'}\frac{\partial t'}{\partial x^{b}} + \frac{\partial x^{a}}{\partial x^{a'}}\frac{\partial x^{a'}}{\partial x^{b}}
\]
and since $t' = t'(t)$, $\partial t'/\partial x^{b} = 0$, hence we reached an important identity of this coordinate transformation
\begin{equation}
\frac{\partial x^{a}}{\partial x^{a'}}\frac{\partial x^{a'}}{\partial x^{b}} = \delta^{a}_{b}
\end{equation}
and equivalently
\begin{equation}
\frac{\partial x^{a'}}{\partial x^{a}}\frac{\partial x^{a}}{\partial x^{b'}} = \delta^{a'}_{b'}
\end{equation}
The spacetime metric transforms regularly under this transformation
\begin{align}\label{spacetime metric}
^{(4)}g_{\mu'\nu'} &=~^{(4)}g_{\mu\nu}\frac{\partial x^{\mu}}{\partial x^{\mu'}}\frac{\partial x^{\nu}}{\partial x^{\nu'}}\notag\\
& = (g_{ab}X^{a}_{\mu}X^{b}_{\nu} - n_{\mu}n_{\nu})\frac{\partial x^{\mu}}{\partial x^{\mu'}}\frac{\partial x^{\nu}}{\partial x^{\nu'}}
\end{align}
By substituting the indices $\mu'$ and $\nu'$ for $a'$ and $b'$ respectively in Eq.~\ref{spacetime metric}, we can obtain the transformation law of the spatial metric
\begin{align}
g_{a'b'} & = (g_{ab}X^{a}_{\mu}X^{b}_{\nu} - n_{\mu}n_{\nu})\frac{\partial x^{\mu}}{\partial x^{a'}}\frac{\partial x^{\nu}}{\partial x^{b'}}\notag\\
&= g_{ab}(\frac{\partial x^{a}}{\partial x^{a'}} + \beta^{a}\frac{\partial t}{\partial x^{a'}})(\frac{\partial x^{b}}{\partial x^{b'}} + \beta^{b}\frac{\partial t}{\partial x^{b'}}) - \alpha^2\frac{\partial t}{\partial x^{a'}}\frac{\partial t}{\partial x^{b'}}\notag\\
& = g_{ab}\frac{\partial x^{a}}{\partial x^{a'}}\frac{\partial x^{b}}{\partial x^{b'}}
\end{align}
According to the result above, the spatial metric $g_{ab}$ transforms as a type $0 \choose 2$ tensor under the spatial diffeomorphism and a scalar under the time reparameterization.

 
And its inverse transforms as
\begin{equation}
g^{a'b'} = g^{ab}\frac{\partial x^{a'}}{\partial x^{a}}\frac{\partial x^{b'}}{\partial x^{b}}\label{spatial metric}
\end{equation}
which is a type $2 \choose 0$ tensor under the spatial diffeomorphism and a scalar under the time reparameterization. 


Let $g$ denote the determinant of the spatial metric $g_{ab}$. It transforms as
\begin{equation}
g' = |g_{a'b'}| = |g_{ab}\frac{\partial x^{a}}{\partial x^{a'}}\frac{\partial x^{b}}{\partial x^{b'}}| = |g_{ab}||\frac{\partial x^{a}}{\partial x^{a'}}||\frac{\partial x^{b}}{\partial x^{b'}}| = g|\frac{\partial x}{\partial x'}|^{2}
\end{equation}
where $|\partial x/\partial x'|$ stands for the Jacobian matrix determinant of this coordinate transformation. This shows that $g$ transforms as a weight +2 density under the spatial diffeomorphism and a scalar under the time reparameterization. 
 
Besides from the spatial metric, the lapse $\alpha$ and the shift vector $\beta^{a}$ are two significant variables in the 3 + 1 splitting scheme. 

From equations [\ref{normal covector}-\ref{spacetime metric 3 + 1}], we have that
\begin{align}
^{(4)}g_{a0} & = \beta_{a}\label{lapse}\\
^{(4)}g_{00} & = -\alpha^2 + g_{ab}\beta^{a}\beta^{b}\label{shift}
\end{align}
According to Eq.~\ref{lapse} and Eq.~\ref{spacetime metric}, we have
\begin{align}
\beta_{a'} & = (g_{ab}X^{a}_{\mu}X^{b}_{\nu} - n_{\mu}n_{\nu})\frac{\partial x^{\mu}}{\partial x^{a'}}\frac{\partial x^{\nu}}{\partial t'}\notag\\
& = g_{ab}X^{a}_{\mu}X^{b}_{\nu}\delta_{c}^{\mu}\frac{\partial x^{c}}{\partial x^{a'}}\frac{\partial x^{\nu}}{\partial t'}\notag\\
& = g_{ab}(\delta^{b}_{\nu} + \beta^{b}\delta^{0}_{\nu})\frac{\partial x^{a}}{\partial x^{a'}}\frac{\partial x^{\nu}}{\partial t'}\notag\\
& = \beta_{a}\frac{\partial x^{a}}{\partial x^{a'}}\frac{\partial t}{\partial t'} + g_{ab}\frac{\partial x^{a}}{\partial x^{a'}}\frac{\partial x^{b}}{\partial t'}\label{shift covector transform}
\end{align}
Use the results from Eq.~\ref{spatial metric} and Eq.~\ref{shift covector transform}, we can easily derive the transform rule of the shift vector
\begin{align}
\beta^{a'} & = \beta_{b'}g^{a'b'}\notag\\
& = \left(\beta_{b}\frac{\partial x^{b}}{\partial x^{b'}}\frac{\partial t}{\partial t'} + g_{cd}\frac{\partial x^{d}}{\partial x^{b'}}\frac{\partial x^{c}}{\partial t'}\right)g^{ab}\frac{\partial x^{a'}}{\partial x^{a}}\frac{\partial x^{b'}}{\partial x^{b}}\notag\\
& = \beta^{a}\frac{\partial x^{a'}}{\partial x^{a}}\frac{\partial t}{\partial t'} + \frac{\partial x^{a'}}{\partial x^{a}}\frac{\partial x^{a}}{\partial t'}\label{shift vector transform}
\end{align}
From the calculation above we can tell that both the shift vector $\beta^{a}$ and the shift covector $\beta_{a}$ do not transform tensorially under this coordinate transformation and we will come back to this issue in the next section. 

Use the result from Eq.~\ref{shift vector transform} along with Eq.~\ref{lapse} and Eq.~\ref{spacetime metric}, we can derive the transform rule of the lapse function. 
\begin{align*}
\alpha'^{2} & = - (g_{ab}X^{a}_{\mu}X^{b}_{\nu} - n_{\mu}n_{\nu})\frac{\partial x^{\mu}}{\partial t'}\frac{\partial x^{\nu}}{\partial t'} + g_{a'b'}\beta^{a'}\beta^{b'}\notag\\ 
& = \alpha^2\left(\frac{\partial t}{\partial t'}\right)^{2}
\end{align*}
Therefore we have
\begin{equation}
\alpha' = \alpha\frac{\partial t}{\partial t'}\label{lapse transform}
\end{equation}
and hence the lapse function transforms as a scalar under the spatial diffeomorphism and a weight +1 density under the time reparameterization. 
 
Now we have investigated the transform rule of the terms without any derivatives. However, since the 1 + log slicing condition and Gamma-driver shift equation are both first order partial differentiation equations, we still need to learn how partial derivative terms transform under the foliation preserving coordinate transformation. 

In order to study how partial derivatives transform, let's review the chain rule for this diffeomorphism briefly. The chain rule in spacetime domain states as below
\begin{equation*}
\partial_{\mu'} = \frac{\partial x^{\mu}}{\partial x^{\mu'}}\partial_{\mu}
\end{equation*}
Therefore, if $\mu' = t'$, we have that
\begin{equation}\label{time derivative}
\partial_{t'} = \frac{\partial x^{c}}{\partial t'}\partial_{c} + \frac{\partial t}{\partial t'}\partial_{t}
\end{equation}
and if $\mu' = a'$, since $\partial t/\partial x^{a'} = 0$, we have
\begin{equation}\label{spatial derivative}
\partial_{a'} = \frac{\partial x^{c}}{\partial x^{a'}}\partial_{c}
\end{equation}
From Eq.~\ref{time derivative} and Eq.~\ref{spatial derivative} we can tell that the spatial derivative transforms just like a type $0 \choose 1$ tensor without introducing any extra terms. On the other hand, in the time derivative transformations, there are one extra term $(\partial t/\partial t')\partial_{t}$ involved, since $\partial t/\partial t' \ne 0$. According to this analysis, the spatial Christoffel symbol transforms as the spacetime Christoffel symbol, since it is only constructed by the spatial metric and its spatial derivatives. 
\begin{equation}
\Gamma^{a'}_{~b'c'} = \Gamma^{a}_{~bc}\frac{\partial x^{a'}}{\partial x^{a}}\frac{\partial x^{b}}{\partial x^{b'}}\frac{\partial x^{c}}{\partial x^{c'}} + \frac{\partial x^{a'}}{\partial x^{d}}\frac{\partial^{2} x^{d}}{\partial x^{b'}\partial x^{c'}}
\end{equation}

Furthermore, the spatial covariant derivative of a tensor also transforms as a tensor. Specifically, the spatial covariant derivative of the shift covector transforms as
\begin{align}
D_{a'}\beta_{b'} &= D_{a'}\left(\beta^{b}\frac{\partial x^{b}}{\partial x^{b'}}\frac{\partial t}{\partial t'}\right) + D_{a'}\left(g_{bc}\frac{\partial x^{b}}{\partial x^{b'}}\frac{\partial x^{c}}{\partial t'}\right)\notag\\
& = \frac{\partial t}{\partial t'}D_{a'}\left(\beta_{b}\frac{\partial x^{b}}{\partial x^{b'}}\right) + D_{a'}\left(g_{b'c'}\frac{\partial x^{c'}}{\partial x^{c}}\frac{\partial x^{c}}{\partial t'}\right)\notag\\
& = D_{a}\beta^{b}\frac{\partial x^{a}}{\partial x^{a'}}\frac{\partial x^{b}}{\partial x^{b'}}\frac{\partial t}{\partial t'} + g_{b'c'}D_{a'}\left(\frac{\partial x^{c'}}{\partial x^{c}}\frac{\partial x^{c}}{\partial t'}\right)
\end{align} 
Now let's examine the time derivative of the spatial metric
\begin{align}
\partial_{t'}g_{a'b'} & = (\partial_{t'}g_{ab})\frac{\partial x^{a}}{\partial x^{a'}}\frac{\partial x^{b}}{\partial x^{b'}} + g_{ab}\partial_{t'}\left(\frac{\partial x^{a}}{\partial x^{a'}}\frac{\partial x^{b}}{\partial x^{b'}}\right)\notag\\
& =  (\partial_{t}g_{ab})\frac{\partial t}{\partial t'}\frac{\partial x^{a}}{\partial x^{a'}}\frac{\partial x^{b}}{\partial x^{b'}} + (\partial_{c}g_{ab})\frac{\partial x^{c}}{\partial t'}\frac{\partial x^{a}}{\partial x^{a'}}\frac{\partial x^{b}}{\partial x^{b'}} + g_{ab}\partial_{t'}\left(\frac{\partial x^{a}}{\partial x^{a'}}\frac{\partial x^{b}}{\partial x^{b'}}\right)
\end{align}
As a result, the extrinsic curvature transforms as a type $0 \choose 2$ tensor under the spatial diffeomorphism and a scalar under the time reparameterization, we omitted the tedious calculation and list the result below
\begin{align}
K_{a'b'} & = -\frac{1}{2\alpha'}\left(\partial_{t'}g_{a'b'} - D_{a'}\beta_{b'} - D_{b'}\beta_{a'}\right)\notag\\
& = K_{ab}\frac{\partial x^{a}}{\partial x^{a'}}\frac{\partial x^{b}}{\partial x^{b'}}
\end{align}
The spatial partial derivative of $\alpha$, which appears in the advection term of the 1 + log slicing equation, transforms as 
\begin{align}
\partial_{a'}\alpha' & = \frac{\partial x^{c}}{\partial x^{a'}}\partial_{c}\left(\alpha \frac{\partial t}{\partial t'}\right)\\
& = \frac{\partial t}{\partial t'}\frac{\partial x^{c}}{\partial x^{a'}}\partial_{c}\alpha
\end{align}
which behaves as a type $0 \choose 1$ tensor under spatial diffeomorphism and a weight +1 density under time reparameterization.

Due to the extra term introduced by time derivative transformation, the time derivative of $\alpha$ raises a more interesting case, 
\begin{align}
\partial_{t'}\alpha' & = \partial_{t'}\left(\alpha\frac{\partial t}{\partial t'}\right)\notag\\
& = \frac{\partial t}{\partial t'}\partial_{t'}\alpha + \alpha\frac{\partial^{2}t}{\partial {t'}^{2}}\notag\\
& = \partial_{t}\alpha\left(\frac{\partial t}{\partial t'}\right)^{2} + \partial_{c}\alpha\frac{\partial x^{c}}{\partial t'}\frac{\partial t}{\partial t'} + \alpha\frac{\partial^{2}t}{\partial {t'}^{2}}
\end{align}

Finally the time derivative of $\beta^{a}$ transforms as
\begin{align}
\partial_{t'}\beta^{a'} & = \partial_{t'}\left(\beta^{a}\frac{\partial x^{a'}}{\partial x^{a}}\frac{\partial t}{\partial t'} + \frac{\partial x^{a'}}{\partial x^{a}}\frac{\partial x^{a}}{\partial t'}\right)\notag\\
& = \frac{\partial t}{\partial t'}\partial_{t}\left(\beta^{a}\frac{\partial x^{a'}}{\partial x^{a}}\frac{\partial t}{\partial t'} + \frac{\partial x^{a'}}{\partial x^{a}}\frac{\partial x^{a}}{\partial t'}\right) + \frac{\partial x^{c}}{\partial t'}\partial_{c}\left(\beta^{a}\frac{\partial x^{a'}}{\partial x^{a}}\frac{\partial t}{\partial t'} + \frac{\partial x^{a'}}{\partial x^{a}}\frac{\partial x^{a}}{\partial t'}\right)\notag\\
& = \partial_{t}\beta^{a}\frac{\partial x^{a'}}{\partial x^{a}}\left(\frac{\partial t}{\partial t'}\right)^{2} + \beta^{a}\frac{\partial t}{\partial t'}\partial_{t}\left(\frac{\partial x^{a'}}{\partial x^{a}}\frac{\partial t}{\partial t'}\right) + \frac{\partial t}{\partial t'}\partial_{t}\left(\frac{\partial x^{a'}}{\partial x^{a}}\frac{\partial x^{a}}{\partial t'}\right)\notag\\
& + \frac{\partial x^{c}}{\partial t'}\partial_{c}\left(\beta^{a}\frac{\partial x^{a'}}{\partial x^{a}}\frac{\partial t}{\partial t'} + \frac{\partial x^{a'}}{\partial x^{a}}\frac{\partial x^{a}}{\partial t'}\right)
\end{align}
\section{Tensorial Variables}
In Sec.~\ref{transform}, we listed the transform rules of all the important 3 + 1 variables and as one can easily notice that some of them do not transform tensorially. In order to solve this, we introduce the backgrond metric denoted as ${\bar g}_{\mu\nu}$. The background metric was initially introduced to make the Christoffel symbols transform as a tensor and it is usually defined as a trivial metric, i.e., ${\bar g}_{\mu\nu} = diag(-1,1,1,1)$. Therefore, we have ${\bar g} = -1, $${\bar \Gamma}^{a}_{~bc} = 0$, ${\bar \alpha} = 1$ and ${\bar \beta}^{a} = 0$. 

The background shift vector ${\bar \beta}^{a}$ follows the same transform rule of $\beta^{a}$
\[
{\bar \beta}^{a'} = {\bar \beta}^{a}\frac{\partial x^{a'}}{\partial x^{a}}\frac{\partial t}{\partial t'} + \frac{\partial x^{a'}}{\partial x^{a}}\frac{\partial x^{a}}{\partial t'}
\]
and hence the difference between $\beta^{a}$ and ${\bar \beta}^{a}$ transform as a type $1 \choose 0$ tensor under spatial diffeomorphism and a weight +1 density under time reparameterization. 
\begin{equation}
\Delta \beta^{a'} = \Delta \beta^{a}\frac{\partial x^{a'}}{\partial x^{a}}\frac{\partial t}{\partial t'}
\end{equation}
where $\Delta \beta^{a} \equiv \beta^{a} - {\bar \beta}^{a}$.

In addition, the tensorial form of Christoffel symbol is $\Delta \Gamma^{a}_{~bc} \equiv \Gamma^{a}_{~bc} - {\bar \Gamma}^{a}_{~bc}$ and
\begin{equation}
\Delta \Gamma^{a'}_{~b'c'} = \Delta \Gamma^{a}_{~bc}\frac{\partial x^{a'}}{\partial x^{a}}\frac{\partial x^{b}}{\partial x^{b'}}\frac{\partial x^{c}}{\partial x^{c'}}
\end{equation}
As we can tell, it transforms as a type $1 \choose 2$ tensor under spatial diffeomorphism and a scalar under time reparameterization. 

In order to construct an invariant form of the time derivative of the lapse function $\mathscr{D}_{t}\alpha$ such that it transforms as a scalar under spatial diffeomorphism and a weight +2 density under time reparameterization.
\[
\mathscr{D}_{t'}\alpha' = \mathscr{D}_{t}\alpha\left(\frac{\partial t}{\partial t'}\right)^{2}
\]
We define $\mathscr{D}_{t}\alpha$ as
\begin{equation}
\mathscr{D}_{t}\alpha \equiv \partial_{t}\alpha - \beta^{c}\partial_{c}\alpha - \frac{\alpha}{{\bar \alpha}}(\partial_{t}{\bar \alpha} - {\bar \beta}^{c}\partial_{c}{\bar \alpha})
\end{equation}
and check how it transforms
\begin{align}
\mathscr{D}_{t'}\alpha' & = \partial_{t'}\alpha' - \beta^{c'}\partial_{c'}\alpha' - \frac{\alpha'}{{\bar \alpha}'}(\partial_{t'}{\bar \alpha}' - {\bar \beta}^{c'}\partial_{c'}{\bar \alpha}')\notag\\
& = \partial_{t}\alpha\left(\frac{\partial t}{\partial t'}\right)^{2} + \partial_{c}\alpha\frac{\partial x^{c}}{\partial t'}\frac{\partial t}{\partial t'} + \alpha\frac{\partial^{2}t}{\partial {t'}^{2}} - \left(\beta^{c}\frac{\partial x^{c'}}{\partial x^{c}}\frac{\partial t}{\partial t'} + \frac{\partial x^{c'}}{\partial x^{c}}\frac{\partial x^{c}}{\partial t'}\right)\frac{\partial t}{\partial t'}\frac{\partial x^{d}}{\partial x^{c'}}\partial_{d}\alpha\notag\\
& - \frac{\alpha}{{\bar \alpha}}\left[\partial_{t}{\bar \alpha}\left(\frac{\partial t}{\partial t'}\right)^{2} + \partial_{c}{\bar \alpha}\frac{\partial x^{c}}{\partial t'}\frac{\partial t}{\partial t'} + {\bar \alpha}\frac{\partial^{2}t}{\partial {t'}^{2}} - \left({\bar\beta}^{c}\frac{\partial x^{c'}}{\partial x^{c}}\frac{\partial t}{\partial t'} + \frac{\partial x^{c'}}{\partial x^{c}}\frac{\partial x^{c}}{\partial t'}\right)\frac{\partial t}{\partial t'}\frac{\partial x^{d}}{\partial x^{c'}}\partial_{d}{\bar\alpha}\right]\notag\\
& = \mathscr{D}_{t}\alpha\left(\frac{\partial t}{\partial t'}\right)^{2}
\end{align}
For formulating the invariant form of the shift vector's time derivative, $\partial_{t}\Delta \beta^{a}$ would b
e a suitable candidate, since $\Delta \beta^{a}$ itself behaves tensorially already. Let's try to transform $\partial_{t} \Delta \beta^{a}$
\begin{align}\label{time derivative delta beta} 
\partial_{t'}\Delta \beta^{a'} & = \left(\partial_{t'}\Delta \beta^{a}\right)\frac{\partial x^{a'}}{\partial x^{a}}\frac{\partial t}{\partial t'} + \Delta \beta^{a}\partial_{t'}\left(\frac{\partial x^{a'}}{\partial x^{a}}\frac{\partial t}{\partial t'}\right)\notag\\
& = \partial_{t}\Delta \beta^{a}\frac{\partial x^{a'}}{\partial x^{a}}\left(\frac{\partial t}{\partial t'}\right)^{2} + \partial_{c}\Delta \beta^{a}\frac{\partial x^{c}}{\partial t'}\frac{\partial x^{a'}}{\partial x^{a}}\frac{\partial t}{\partial t'}\notag\\
& + \Delta \beta^{a}\frac{\partial x^{a'}}{\partial x^{a}}\frac{\partial^{2}t}{\partial t'^{2}} + \Delta \beta^{a}\frac{\partial^{2} x^{a'}}{\partial x^{c}\partial x^{a}}\frac{\partial x^{c}}{\partial t'}\frac{\partial t}{\partial t'} + \Delta \beta^{a}\frac{\partial^{2}x^{a'}}{\partial t\partial x^{a}}\left(\frac{\partial t}{\partial t'}\right)^{2}
\end{align}
As we expected, there are several extra terms in Eq.~\ref{time derivative delta beta} prevent $\partial_{t}\Delta\beta^{a}$ from being a tensor density. 

By observing the transform rules of $\partial_{t}\Delta\beta^{a}$, $\beta^{a}$, $D_{a}\beta_{b}$ and $\partial_{a}\alpha$, we propose an ansatz of $\mathscr{D}_{t}\beta^{a}$ such that $\mathscr{D}_{t}\beta^{a}$ transforms as a type $1 \choose 0$ tensor under spatial diffeomorphism and a weight +2 density under time reparameterization, 
\begin{equation}
\mathscr{D}_{t'}\beta^{a'} = \mathscr{D}_{t}\beta^{a}\frac{\partial x^{a'}}{\partial x^{a}}\left(\frac{\partial t}{\partial t'}\right)^{2}
\end{equation}
where $\mathscr{D}_{t}\beta^{a}$ is defined as
\begin{equation}
\mathscr{D}_{t}\beta^{a} \equiv \partial_{t}\Delta \beta^{a} - \frac{\Delta \beta^{a}}{{\bar \alpha}}(\partial_{t}{\bar \alpha} - {\bar \beta}^{a}\partial_{a}{\bar \alpha}) + c_{1}\beta^{b}{\bar D}_{b}\beta^{a} + c_{2}\beta^{b}{\bar D}_{b}{\bar \beta}^{a} + c_{3}{\bar \beta}^{b}{\bar D}_{b}\beta^{a} + c_{4}{\bar \beta}^{b}{\bar D}_{b}{\bar \beta}^{a}
\end{equation}
and $c_{1}$, $c_{2}$, $c_{3}$ and $c_{4}$ are all constants. 

Determining the values of these constants involves tedious calculation, so we skip to the conclusion here. The constants obey the following constraints
\begin{align*}
c_{1} + c_{3} & = -1\\
c_{2} + c_{4} & = 1\\
c_{1} + c_{2} & = 1\\
c_{3} + c_{4} & = -1 
\end{align*}
so that $\mathscr{D}_{t} \beta^{a}$ transforms as a tensor density. 

Therefore, we define $\sigma \equiv c_{1}$ and write the rest of those constants in terms of $\sigma$ to get the final formulation of $\mathscr{D}_{t}\beta^{a}$ as below
\begin{equation}
\mathscr{D}_{t}\beta^{a} = \partial_{t}\Delta \beta^{a} - \frac{\Delta \beta^{a}}{{\bar \alpha}}(\partial_{t}{\bar \alpha} - {\bar \beta}^{a}\partial_{a}{\bar \alpha}) + \sigma\beta^{b}{\bar D}_{b}\beta^{a} + (1-\sigma)\beta^{b}{\bar D}_{b}{\bar \beta}^{a} - (1 + \sigma){\bar \beta}^{b}{\bar D}_{b}\beta^{a} + \sigma{\bar \beta}^{b}{\bar D}_{b}{\bar \beta}^{a}
\end{equation}

\section{Invariant Gauge Condition}\label{gauge}
We studied the transform rules of all the important variables in 3 + 1 splitting mechanism under the foliation preserving coordinate transformation in Sec.~\ref{transform}, now we will use these results to construct the invariant form of the 1 + log slicing condition and Gamma-driver shift equation. 

The regular 1 + log slicing condition reads as
\begin{equation}\label{1 + log slicing}
\partial_{t}\alpha - \beta^{c}\partial_{c}\alpha = -2\alpha K
\end{equation}
After the analysis in Sec.~\ref{transform}, one can notice that neither the right hand side nor the left hand side transforms tensorially. We start by replacing the left hand side by $\mathscr{D}_{t}\alpha$. Since $\mathscr{D}_{t'}\alpha' = \mathscr{D}_{t}\alpha(\partial t/\partial t')^{2}$, in order to balance the right hand side, we modify it to be $-2\alpha {\bar \alpha}K$. After this modification, we have
\begin{align*}
\mathscr{D}_{t'}\alpha' & = \mathscr{D}_{t}\alpha\left(\frac{\partial t}{\partial t'}\right)^{2}\\
& = -2\alpha{\bar \alpha}K\left(\frac{\partial t}{\partial t'}\right)^{2}\\
& = -2\left(\alpha\frac{\partial t}{\partial t'}\right)\left({\bar \alpha}\frac{\partial t}{\partial t'}\right)K\\
& = -2\alpha'{\bar \alpha}'K'
\end{align*}
Therefore, we have reached the invariant form of 1 + log slicing condition as below
\begin{equation}\label{invariant 1 + log slicing}
\mathscr{D}_{t}\alpha = -2\alpha{\bar \alpha}K
\end{equation}

We can follow the same strategy to alter the Gamma-driver shift equation, replacing the left hand side by $\mathscr{D}_{t}\beta^{a}$ and setting $\sigma$ to be -1, changing every non-tensorial term on the right hand side to its corresponding tensorial term and balance the transform factor on both sides. Therefore, the regular Gamma-driver shift equation
\begin{equation}\label{Gamma-driver shift}
{\dot \beta}^{a} - \beta^{b}\partial_{b}\beta^{a} = \frac{3}{4}\sqrt{g}^{2/3}(\Gamma^{a}_{~bc}g^{bc} + \frac{1}{3}g^{ab}\Gamma^{c}_{~bc}) - \eta \beta^{a} 
\end{equation}
becomes
\begin{equation}\label{invariant Gamma-driver shift}
\mathscr{D}_{t}\beta^{a} = \frac{3}{4}\sqrt{\frac{g}{{\bar g}}}^{2/3}{\bar \alpha}^{2}(\Delta \Gamma^{a}_{~bc}g^{bc} + \frac{1}{3}g^{ab}\Delta\Gamma^{c}_{~bc}) - \eta {\bar \alpha}\Delta\beta^{a}
\end{equation}
Writing down these invariant gauge conditions creates more possibilities for numerical simulation. When the need of evolving the system in a different coordinates system rises, one can just use the following equations as gauge conditions
\begin{align*}
\mathscr{D}_{t'}\alpha' & = -2\alpha'{\bar \alpha}'K'\\
\mathscr{D}_{t'}\beta^{a'} & = \frac{3}{4}\sqrt{\frac{g'}{{\bar g}'}}^{2/3}{\bar \alpha}'^{2}(\Delta \Gamma^{a'}_{~b'c'}g^{b'c'} + \frac{1}{3}g^{a'b'}\Delta\Gamma^{c'}_{~b'c'}) - \eta {\bar \alpha}'\Delta\beta^{a'}
\end{align*}
while the transformed background metric terms are associated with the original terms as below
\begin{align}
{\bar g}_{a'b'} & = {\bar g}_{ab}\frac{\partial x^{a}}{\partial x^{a'}}\frac{\partial x^{b}}{\partial x^{b'}}\\
{\bar g'} & = {\bar g}|\frac{\partial x}{\partial x'}|^{2}\\
{\bar \alpha}' & = {\bar \alpha}\frac{\partial t}{\partial t'}\\
{\bar \beta}^{a'} & = {\bar \beta}^{a}\frac{\partial x^{a'}}{\partial x^{a}}\frac{\partial t}{\partial t'} + \frac{\partial x^{a'}}{\partial x^{c}}\frac{\partial x^{c}}{\partial t'}\label{background shift}\\
{\bar \Gamma}^{a'}_{~b'c'} & = {\bar \Gamma}^{a}_{~bc}\frac{\partial x^{a'}}{\partial x^{a}}\frac{\partial x^{b}}{\partial x^{b'}}\frac{\partial x^{c}}{\partial x^{c'}} + \frac{\partial x^{a'}}{\partial x^{d}}\frac{\partial x^{d}}{\partial x^{b'}\partial x^{c'}}
\end{align}

We can also achieve the transformed gauge conditions from the regular gauge conditions. All we need to do is using the transform rules we derived above to write the non-transformed variables in terms of the transformed terms. And we argue as following that these two methods should yield identical transformed gauge conditions. 

Assume we have some variables $v$. Under a change of coordinates, they transform to $v' = T(v)$ where $T$ is some function. We also introduce a 
set of background fields that transform in the same way; thus $\bar v' = T(\bar v)$. The gauge conditions in covariant form are $F^A(v,\bar v) = 0$, 
where the index $A$ ranges over the number of conditions. Since the gauge conditions are covariant, we have 
\begin{equation}
	F^A(v',\bar v') = \xi^A_B F^B(v,\bar v)
\end{equation}
where $\xi^A_B$ is an invertible matrix. That is, 
\begin{equation}\label{invariant equivalence}
	 F^A(v,\bar v) = 0 \Longleftrightarrow F^A(v',\bar v') = 0
\end{equation}
In the original coordinate system, the gauge condition is $F^A(v,\bar v) = 0$ and in the transformed coordinate system, the 
gauge condition is $F^A(v',\bar v') = 0$.

Now we want to consider a particular background. Assume in the original coordinate system, the background fields take the values $\bar v = \bar c$. The gauge 
conditions in original coordinates are $F^A(v,\bar c) = 0$. (This is the non--covariant version of the gauge conditions.) As we discussed above, we have two ways to achieve the transformed gauge conditions. One way: 
\begin{itemize}
	\item Start with the gauge written in covariant form, $F^A(v,\bar v) = 0$.
	\item Change to transformed coordinates, where the gauge is $F^A(v',\bar v') = 0$.
	\item Set the background to $\bar v = \bar c$. But $\bar v' = T(\bar v)$ so that $\bar v' = T(\bar c)$. 
	\item Then the gauge conditions in transformed coordinates read $F^A(v',T(\bar c)) = 0$. 
\end{itemize}
Another way: 
\begin{itemize}
	\item Start with the gauge written in covariant form, $F^A(v,\bar v) = 0$.
	\item Set the background to $\bar v = \bar c$, so the gauge takes the non--covariant form $F^A(v,\bar c) = 0$. 
	\item Now use the transformation rules to replace the original fields with transformed fields; that is, use $v = T^{-1}(v')$. 
	The gauge becomes $F^A(T^{-1}(v'),\bar c) = 0$. 
\end{itemize}
These two results are equivalent. That is, 
\begin{equation}
	F^A(T^{-1}(v'),\bar c) = 0 \Longleftrightarrow F^A(v',T(\bar c)) = 0
\end{equation}
This is guaranteed by the Eq.~\ref{invariant equivalence} above. In Eq.~\ref{invariant equivalence}, just let $v = T^{-1}(v')$ and $\bar v = \bar c$, which implies $\bar v' = T(\bar c)$.
 
We will explain how this scheme works in detail with a specific example in the next section. 

\section{Time Dependent Spatial Rotation Coordinate Transformation}\label{example}
In this section, we examine a specific kind of coordinate transformation, the time dependent spatial rotation coordinate transformation, and demonstrate how the invariant gauge conditions work in action. 

In this coordinate transformation, we keep the $t$ coordinate intact while let the spatial coordinate rotates around the $z$ axis at a constant angular speed $\omega$. Hence the transformed coordinates can be expressed in terms of the original terms as below
\begin{align}
t' & = t\notag\\
x' & = x \cos(\omega t) + y\sin(\omega t)\notag\\
y' & = -x\sin(\omega t) + y\cos(\omega t)\notag\\
z' & = z
\end{align}
and equivalently, 
\begin{align}
t & = t'\notag\\
x & = x'\cos(\omega t') - y'\sin(\omega t')\notag\\
y & = x'\sin(\omega t') + y'\cos(\omega t')\notag\\
z & = z'
\end{align}
We also assume that the background metric is trivial
\begin{align}
{\bar g}_{ab} & = diag(1,1,1)\notag\\
{\bar \alpha} & = 1\notag\\
{\bar \beta} & = 0 
\end{align}
Under this transformation, we have $\partial t/\partial t' = 1$, therefore ${\bar \alpha}' = {\bar \alpha}$ and hence the invariant time derivative of $\alpha$ will collapse to
\begin{equation}
\mathscr{D}_{t}\alpha = \partial_{t}\alpha - \beta^{a}\partial_{a}\alpha
\end{equation}
Then the invariant 1 + log slicing condition under this transformation becomes
\begin{equation}\label{transformed 1 + log slicing}
\partial_{t'}\alpha' - \beta^{a'}\partial_{a'}\alpha' = -2\alpha'K'
\end{equation}
One would argue that this equation is identical to the regular 1 + log slicing equation. This is only true under this specific coordinate transformation, especially when we have $\partial t/\partial t' = 1$. One will see the difference in the Gamma-driver shift equation discussed below. 

According to Eq.~\ref{background shift} and the choice that ${\bar \beta}^{a} = 0$, we have that
\[
{\bar \beta}^{a'} = \frac{\partial x^{a'}}{\partial x^{a}}\frac{\partial x^{a}}{\partial t'}
\]
Therefore, for each component, we have
\begin{align*}
{\bar \beta}^{1'} & = -\omega y'\\
{\bar \beta}^{2'} & = \omega x'\\
{\bar \beta}^{3'} & = 0
\end{align*}
Similarly, for the background Christoffel symbol, we have
\[
{\bar \Gamma}^{a'}_{b'c'} = \frac{\partial x^{a'}}{\partial x^{d}}\frac{\partial x^{d}}{\partial x^{b'}\partial x^{c'}}
\]
however, since in this coordinate transformation, $x^{a}$ is a first order function of $x^{a'}$, ${\bar \Gamma}^{a'}_{b'c'}$ vanishes. 

Also, it is easy to check that $|\partial x/\partial x'| = 1$, this result yields that $g' = g = 1$. According to these results above, the transformed Gamma-driver Shift equation becomes
\begin{equation}
\partial_{t'}\beta^{a'} - \beta^{b'}\partial_{b'}\beta^{a'} + 2\beta^{b'}\partial_{b'}{\bar \beta}^{a'} - {\bar \beta}^{b'}\partial_{b'}{\bar \beta}^{a'} = \frac{3}{4}\sqrt{g'}^{2/3}\left(\Gamma^{a'}_{~b'c'}g^{b'c'} + \frac{1}{3}g^{a'b'}\Gamma^{c'}_{~b'c'}\right) - \eta \Delta \beta^{a'}
\end{equation}
and for each component, we have
\begin{align}
\partial_{t'}\beta^{1'} - \beta^{b'}\partial_{b'}\beta^{1'} - 2\omega\beta^{2'} + \omega^{2}x' &= \frac{3}{4}\sqrt{g'}^{2/3}\left(\Gamma^{1'}_{~b'c'}g^{b'c'} + \frac{1}{3}g^{1'b'}\Gamma^{c'}_{~b'c'}\right) - \eta (\beta^{1'} + \omega y')\label{transformed Gamma-driver shift x}\\
\partial_{t'}\beta^{2'} - \beta^{b'}\partial_{b'}\beta^{2'} + 2\omega\beta^{1'} + \omega^{2}y' &= \frac{3}{4}\sqrt{g'}^{2/3}\left(\Gamma^{2'}_{~b'c'}g^{b'c'} + \frac{1}{3}g^{2'b'}\Gamma^{c'}_{~b'c'}\right) - \eta (\beta^{2'} - \omega x')\label{transformed Gamma-driver shift y}\\
\partial_{t'}\beta^{3'} - \beta^{b'}\partial_{b'}\beta^{3'} & = \frac{3}{4}\sqrt{g'}^{2/3}\left(\Gamma^{3'}_{~b'c'}g^{b'c'} + \frac{1}{3}g^{3'b'}\Gamma^{c'}_{~b'c'}\right) - \eta \beta^{3'}\label{transformed Gamma-driver shift z}
\end{align}
To verify the transformed gauge conditions(Eq.~\ref{transformed 1 + log slicing} and equations [\ref{transformed Gamma-driver shift x}-\ref{transformed Gamma-driver shift y}]) obtained by the invariant gauge conditions, we try to write down the transformed gauge conditions by transforming the variables in the regular gauge conditions explicitly and compare with the equations we already have. 

According to the transform rules we obtained in \ref{transform}, let's write the untransformed variables in terms of the transformed variables. 
\begin{align}
\alpha & = \alpha'\frac{\partial t'}{\partial t} = \alpha'\label{alpha to alpha prime}\\
\beta^{a} & = \left(\beta^{a'} - \frac{\partial x^{a'}}{\partial x^{b}}\frac{\partial x^{b}}{\partial t'}\right)\frac{\partial x^{a}}{\partial x^{a'}}\frac{\partial t'}{\partial t} =   \left(\beta^{a'} - \frac{\partial x^{a'}}{\partial x^{b}}\frac{\partial x^{b}}{\partial t'}\right)\frac{\partial x^{a}}{\partial x^{a'}}\label{beta to beta prime}\\
g_{ab} & = g_{a'b'}\frac{\partial x^{a'}}{\partial x^{a}}\frac{\partial x^{b'}}{\partial x^{b}}\label{metric to metric prime}\\
g & = g'\left(\frac{\partial x'}{\partial x}\right)^{2} = g'\label{det to det prime}\\
\Gamma^{a}_{~bc} & = \left(\Gamma^{a'}_{~b'c'} - \frac{\partial x^{a'}}{\partial x^{d}}\frac{\partial x^{d}}{\partial x^{b'}\partial x^{c'}}\right)\frac{\partial x^{a}}{\partial x^{a'}}\frac{\partial x^{b'}}{\partial x^{b}}\frac{\partial x^{c'}}{\partial x^{c}} = \Gamma^{a'}_{~b'c'}\frac{\partial x^{a}}{\partial x^{a'}}\frac{\partial x^{b'}}{\partial x^{b}}\frac{\partial x^{c'}}{\partial x^{c}}\label{gamma to gamma prime}\\
K & = K'\label{K to K prime}\\
\end{align}
Plug Eq.~\ref{alpha to alpha prime}, Eq.~\ref{beta to beta prime} and Eq.~\ref{K to K prime} into Eq.~\ref{1 + log slicing}, we have
\begin{align}
\partial_{t}\alpha' - \left(\beta^{c'} - \frac{\partial x^{c'}}{\partial x^{d}}\frac{\partial x^{d}}{\partial t'}\right)\frac{\partial x^{c}}{\partial x^{c'}}\partial_{c}\alpha' & = -2\alpha' K'\notag\\
\partial_{t'}\alpha' + \frac{\partial x^{c'}}{\partial t}\partial_{c'}\alpha' - \beta^{c'}\partial_{c'}\alpha' + \frac{\partial x^{c'}}{\partial x^{d}}\frac{\partial x^{d}}{\partial t'}\partial_{c'}\alpha' & = -2\alpha' K'\notag\\
\partial_{t'}\alpha' + \frac{\partial x^{c'}}{\partial t}\partial_{c'}\alpha' - \beta^{c'}\partial_{c'}\alpha' - \frac{\partial x^{c'}}{\partial t}\frac{\partial t}{\partial t'}\partial_{c'}\alpha' & = -2\alpha' K'\notag\\
\partial_{t'}\alpha'  - \beta^{c'}\partial_{c'}\alpha' & = -2\alpha' K'
\end{align}
Hereby we reached the identical equation as Eq.~\ref{transformed 1 + log slicing}.

Now let's plug equations [\ref{beta to beta prime} - \ref{gamma to gamma prime}] into Eq.~\ref{Gamma-driver shift} and write the regular Gamma-driver shift equation in terms of the transformed variables. 

Since this is a long equation, let's first examine the left hand side, 
\begin{align}
& \partial_{t}\left[\left(\beta^{a'} - \frac{\partial x^{a'}}{\partial x^{c}}\frac{\partial x^{c}}{\partial t'}\right)\frac{\partial x^{a}}{\partial x^{a'}}\right] - \left[\left(\beta^{b'} - \frac{\partial x^{b'}}{\partial x^{c}}\frac{\partial x^{c}}{\partial t'}\right)\frac{\partial x^{b}}{\partial x^{b'}}\right]\partial_{b}\left[\left(\beta^{a'} - \frac{\partial x^{a'}}{\partial x^{c}}\frac{\partial x^{c}}{\partial t'}\right)\frac{\partial x^{a}}{\partial x^{a'}}\right]\notag\\
= & \partial_{t}\left(\beta^{a'}\frac{\partial x^{a}}{\partial x^{a'}} - \frac{\partial x^{a}}{\partial t'}\right) - \left(\beta^{b'} - \frac{\partial x^{b'}}{\partial x^{c}}\frac{\partial x^{c}}{\partial t'}\right)\partial_{b'}\left(\beta^{a'}\frac{\partial x^{a}}{\partial x^{a'}} - \frac{\partial x^{a}}{\partial t'}\right)\notag\\
= & \left(\partial_{t'}\beta^{a'}\right)\frac{\partial x^{a}}{\partial x^{a'}} - \left(\beta^{b'}\partial_{b'}\beta^{a'}\right)\frac{\partial x^{a}}{\partial x^{a'}} + 2\beta^{a'}\left(\partial_{t'}\frac{\partial x^{a}}{\partial x^{a'}}\right) - \frac{\partial^{2}x^{a}}{\partial t'^{2}}\notag\\
= & \left(\partial_{t'}\beta^{a'} - \beta^{b'}\partial_{b'}\beta^{a'} - 2\beta^{b'}\frac{\partial x^{c}}{\partial x^{b'}}\frac{\partial^{2}x^{a'}}{\partial x^{c}\partial t} + \frac{\partial^{2} x^{a'}}{\partial t^{2}} + 2\frac{\partial^{2} x^{a'}}{\partial t\partial x^{c}}\frac{\partial x^{c}}{\partial t'}\right)\frac{\partial x^{a}}{\partial x^{a'}}
\end{align}
Then the right hand side is
\begin{align}
\frac{3}{4}\sqrt{g'}^{2/3}\left(\Gamma^{a'}_{~b'c'}g^{b'c'} + \frac{1}{3}g^{a'b'}\Gamma^{c'}_{~b'c'}\right)\frac{\partial x^{a}}{\partial x^{a'}} - \eta\left(\beta^{a'} - \frac{\partial x^{a'}}{\partial x^{b}}\frac{\partial x^{b}}{\partial t'}\right)\frac{\partial x^{a}}{\partial x^{a'}}
\end{align}
Multiply $\partial x^{a'}/\partial x^{a}$ on both sides of the equation, we have
\begin{equation}
\partial_{t'}\beta^{a'} - \beta^{b'}\partial_{b'}\beta^{a'} - 2\beta^{b'}\frac{\partial x^{c}}{\partial x^{b'}}\frac{\partial^{2}x^{a'}}{\partial x^{c}\partial t} + \frac{\partial^{2} x^{a'}}{\partial t^{2}} +  2\frac{\partial^{2} x^{a'}}{\partial t\partial x^{c}}\frac{\partial x^{c}}{\partial t'}= \frac{3}{4}\sqrt{g'}^{2/3}\left(\Gamma^{a'}_{~b'c'}g^{b'c'} + \frac{1}{3}g^{a'b'}\Gamma^{c'}_{~b'c'}\right) - \eta\left(\beta^{a'} - \frac{\partial x^{a'}}{\partial x^{b}}\frac{\partial x^{b}}{\partial t'}\right)\label{inverted Gamma-driver shift}
\end{equation}
and one can easily check that Eq.~\ref{inverted Gamma-driver shift} is equivalent to equations [\ref{transformed Gamma-driver shift x} - \ref{transformed Gamma-driver shift z}].
\end{document}
