\documentclass[letterpaper,nofootinbib,prd,amsmath,onecolumn]{revtex4-1}

\usepackage{amsmath}
\usepackage{amsthm}
\usepackage{graphicx}
\usepackage{float}
%\usepackage{listings}
%\usepackage{pdfpages}
\usepackage{courier}
\usepackage{color}
%\usepackage{mathrsfs}

\begin{document}
\special{papersize=8.5in,11in}

\title{Covariant Gauge Conditions for Foliation Preserving Coordinate Transformations}
\author{Meng Cao, J.~David Brown}
\affiliation{Department of Physics, North Carolina State University, Raleigh, NC 27695 USA}

\begin{abstract}
The transformation rules of 3 + 1 splitting variables, such as the lapse function, shift vector, extrinsic curvature and etc., under a foliation preserving coordinate transformation are presented. The covariant forms of time derivative of the lapse function and time derivative of the shift vector are given. The invariant forms of 1 + log slicing condition and Gamma-driver shift condition are derived from those covariant time derivatives. Utilizing these invariant gauge conditions allows one to evolve the same system with two different coordinate systems and reach the same result. A time dependent spatial rotation coordinate transformation is discussed in details as a specific example of this case.  
\end{abstract}
\maketitle

%%%%%%%%%%%%%%%%%%%%%%%%%%%%%%%%%%%%%%%%
\section{Introduction}
In numerical simulations of Einstein gravity, the (spatial or spacetime) metric is evolved in time using one of several formulations of the Einstein equations, such as the 
Baumgarte--Shapiro--Shibata--Nakamura (BSSN) system \cite{Shibata:1995we,Baumgarte:1998te} or the generalized harmonic (GH) 
system \cite{Friedrich:1985,Garfinkle:2001ni,Pretorius:2006tp}. The result is a description of the spacetime geometry in terms of a {\em foliation}, that is, a splitting of spacetime into 
constant time slices. 
Each of these evolution systems requires the specification of gauge conditions which, directly or indirectly, determine the lapse function and shift vector. 
The lapse and shift tell the proper time separation and coordinate shift between successive time slices. 

The metric on each time slice determines the geometry, which is a coordinate independent object. We can imagine two different numerical simulations that yield the 
same foliation of spacetime and the same geometry, but use different coordinate systems. How are the gauge conditions related for these two simulations? The 
answer is not immediately obvious because the conditions commonly used with BSSN and GH are not generally covariant. 

For BSSN, the most common gauge choice is the ``moving puncture gauge", which consists of 1+log slicing \cite{Bona:1994dr} and the
gamma--driver shift \cite{Alcubierre:2002kk}. Gauge conditions for GH are discussed in Refs.~\cite{Pretorius:2006tp,Szilagyi:2009qz}.  
The main goal of this paper is to rewrite each of the gauge conditions in a form that is invariant under coordinate transformations that 
preserve the foliation of spacetime into constant time slices. We refer to these as ``foliation--preserving transformations."

In Sec.~\ref{covariance} we discuss general covariance and foliation--preserving transformations, and describe our main goal in more detail. 
In Sec.~\ref{transform}, we review the 3+1 splitting of the spacetime metric and define the spatial metric, lapse function, shift vector and extrinsic curvature. 
We then derive the transformation rules for all 3+1 variables and their time derivatives. 
Background fields are introduced in Sec.~\ref{tensorial}. These are used to construct covariant combinations of the 3+1 variables and their time derivatives. 
The gauge conditions are discussed in detail in Sec.~\ref{gauge}, and written in covariant form. 
Finally, in Sec.~\ref{example}, we demonstrate how the covariant gauge conditions for BSSN are used in a specific case, namely, the transformation 
from non--rotating to rotating spatial coordinates. 

%The Einstein's equations are in the covariant form under spacetime diffeomorphism, which means the equations are invariant under spacetime coordinate transformations. 3 + 1 splitting mechanism slices the spacetime into a family of spacelike hypersurfaces which is referred as foliations. On each foliation, a timelike parameter t is constant. By breaking the symmetry of time and space domain, the 3 + 1 formulation makes it possible to evolve a general relativity system along with time numerically. On the other hand, although it still preserves the invariant property, it is not manifest in the equations anymore. While constructing numerical simulations, either in ADM formulation \cite{ADM:Witten} or BSSN({\color{red}reference BSSN?}) formulation, people use gauge conditions(such as 1 + log slicing and Gamma-driver shift condition) to evolve the lapse function and shift vector along with time. Unfortunately, these gauge condition equations are neither explicitly nor implicitly covariant. This scheme works sufficiently accurate if the system is evolved in one single coordinate system. However, if the same system is evolved with two different coordinate systems to the same time under this kind of gauge conditions, it is not guaranteed that we can get the same results. 

%This situation leads to our motivation of this paper. We will try to examine the transformation rules of physical variables in 3 + 1 splitting system under a so called ``foliation preserving coordinate transformation'', which includes time reparamerization and time dependent spatial deffeomorphism while the structure of foliations is preserved. Then we will try to write down the covariant form of the gauge conditions under this coordinate transformation and give the guideline of how to implement this into numerical simulation. 

%The paper is organized as follows. First of all, we will present the idea of general covariance and how it is broken in ADM and BSSN formulations in Sec.~\ref{covariance}. 

%%%%%%%%%%%%%%%%%%%%%%%%%%%%%%%%%%%%%%%%%%%%%%%%%%%%%%%%%%%%%%%%%%
\section{General covariance}\label{covariance}
The Einstein field equations take the same 
form in any spacetime coordinate system. We refer to this property as {\em general covariance}.\footnote{The definition of 
general covariance is not entirely uniform throughout the literature \cite{Weinberg:GRbook,Misner:1974qy,Wald:GRbook,Lambourne:GRbook}. Most authors 
define general covariance, as we do, to include the concept of form invariance under changes of coordinates.} 
With the Einstein equations written as\footnote{We use Greek letter indices $\mu$, $\nu$, ... to denote spacetime indices and Latin 
letters $a$, $b$, ... for spatial indices. $\nabla_{\mu}$ is the spacetime covariant derivative while $D_{a}$ is the spatial 
covariant derivative. Newton's constant it $G$.} $G_{\mu\nu} = 8\pi G\, T_{\mu\nu}$, and given the transformation properties 
of the individual terms $G_{\mu\nu}$ and $T_{\mu\nu}$, general covariance is easy to show. To do so, we first observe that 
the Einstein tensor $G_{\mu\nu}$ and the stress--energy tensor $T_{\mu\nu}$ transform as type $0\choose 2$ tensors under changes 
of spacetime coordinates. Thus, under the  
transformation $x^{\mu'} = x^{\mu'}(x^\mu)$, we have
\begin{equation}
	G_{\mu\nu} = \frac{\partial x^{\mu'}}{\partial x^{\mu}} \frac{\partial x^{\nu'}}{\partial x^{\nu}} G_{\mu'\nu'} 
\end{equation}
with a corresponding expression for $T_{\mu\nu}$. By applying the coordinate transformation 
to the Einstein equations we find that the factors 
$(\partial x^{\mu'}/\partial x^{\mu}) (\partial x^{\nu'}/\partial x^{\nu})$
cancel, leaving $G_{\mu'\nu'} = 8\pi G\, T_{\mu'\nu'}$. 

Equations like $G_{\mu\nu} = 8\pi G T_{\mu\nu}$, whose general covariance is trivial to confirm by virtue 
of the transformation properties of the individual terms, will be referred to as {\em manifestly covariant}. 

The Einstein equations expressed in 3+1 form are the Arnowitt--Deser--Misner (ADM) equations \cite{ADM:Witten}. 
We write them here in the form used by Smarr and York \cite{Smarr:York,Smarr:1977uf}:
\begin{subequations}\label{ADMeqns}
\begin{eqnarray}
	\partial_t g_{ab} - {\cal L}_\beta g_{ab} & = & -2\alpha K_{ab} \ , \\
	\partial_t K_{ab} - {\cal L}_\beta K_{ab} & = & \alpha K K_{ab} - 2\alpha K_{ac}K^c_b 
		+ \alpha R_{ab} - D_a D_b\alpha \ .
\end{eqnarray}
\end{subequations}
Matter fields have been set to zero for simplicity.  These equations are expressed in terms of the 
spatial metric $g_{ab}$, extrinsic curvature $K_{ab}$, lapse function $\alpha$ and shift vector $\beta^a$. 
Since the ADM equations hold in any spacetime coordinate system, they are generally covariant. They are not, however, 
manifestly covariant. This is because the ADM equations are written in terms  
fields $g_{ab}$,  $K_{ab}$, $\alpha$ and $\beta^a$, that do not transform as tensors under 
spacetime coordinate transformations. 

Although the ADM equations are not manifestly covariant under changes of {\em spacetime} coordinates, they are 
manifestly covariant under changes of {\em spatial} coordinates. This is because each term in Eqs.~(\ref{ADMeqns})
transforms as a type $0\choose 2$ tensor under spatial coordinate transformations. In fact, we will show that the ADM  
equations are manifestly covariant under a wider class of transformations that preserve the foliation of spacetime 
into spacelike hypersurfaces. These transformations are defined by
\begin{subequations}\label{foliationpreservingtransf}
\begin{eqnarray}
t' & = & t'(t) \ ,\\
x^{a'} & = & x^{a'}(t,x^{a}) \ .
\end{eqnarray}
\end{subequations}
They consist of time reparametrizations (\ref{foliationpreservingtransf}a) and {\em time--dependent} spatial coordinate transformations 
(\ref{foliationpreservingtransf}b). We   refer to Eqs.~(\ref{foliationpreservingtransf}) as ``foliation--preserving transformations". 

The ADM equations provide the foundation for interpreting Einstein's theory as an initial value problem, 
and for solving the Einstein equations numerically. 
The lack of manifest spacetime covariance is not a problem in this context. In fact, in writing the Einstein equations in 3+1 
form, our point of view shifts away from 
spacetime fields to time--dependent fields on space. The manifest covariance of the ADM equations under foliation preserving 
transformations both supports and motivates this point of view. 

Let us consider more carefully the concept of general covariance in the 3+1 context. 
In general relativity, coordinate systems themselves have little meaning apart from certain 
global or topological restrictions that they might impose. For example, if a coordinate $x^3$ has a finite range (for example, $0$ to $2\pi$) 
and we impose periodic boundary conditions on the fields, then the $x^3$ coordinate lines are topologically circles. 
Although we often use names like ``Cartesian coordinates" or ``spherical coordinates", these names do not tell us anything 
about the geometry. They 
are merely a shorthand way of communicating information about the ranges of the coordinates and the topology of the underlying manifold. 

What does have a definite meaning is the comparison of fields using two different coordinate systems. For example, let 
$t,x^{a}$ and $t',x^{a'}$ denote  
two different coordinate systems, related as in Eq.~(\ref{foliationpreservingtransf}). We can carry out two separate numerical simulations, 
evolving the fields in the unprimed and primed coordinate systems, and compare the results to one another.  

How is this comparison done? 
First we prepare initial data with coordinates labeled $x^a$, letting $t_i$ denote the initial time. This is the unprimed initial data. We evolve this data to a final 
time $t_f$ using some set of partial differential equations. Next, we use 
the spatial transformation $x^{a'} = x^{a'}(t_i,x^a)$ and the transformation rules for the fields to transform the unprimed initial data to the primed 
system. This transformed data is the initial data for the evolution in primed coordinates, starting at the initial time $t'_i = t'(t_i)$. 
The differential equations used to evolve the primed data are obtained from the ``unprimed differential equations" using the transformations (\ref{foliationpreservingtransf}) and 
the transformation rules for the fields. If the equations are generally covariant under these transformations, then the primed equations have the same form 
as the unprimed equations. Using the primed evolution equations, we then evolve the primed data to the final 
time $t'_f = t'(t_f)$. The final data in primed coordinates will be related to the final data in unprimed coordinates by the coordinate transformation 
$x^{a'} = x^{a'}(t_f,x^a)$. 

The ADM equations are covariant, so they take the same form in the primed coordinates as in the unprimed coordinates. However, it is now 
well recognized that the ADM equations 
are not suited for numerical applications. Current numerical codes use one of several alterative systems, such as 
BSSN  \cite{Shibata:1995we,Baumgarte:1998te} or  GH \cite{Friedrich:1985,Garfinkle:2001ni}.
The BSSN and GH equations are generally covariant under foliation preserving transformations but, as they are usually written, 
they not manifestly covariant. The lack of manifest covariance stems from the use of non--tensorial fields as independent variables. 

Each of the systems, BSSN and GH, must be supplemented with a set of equations that fix the gauge freedom. 
The gauge conditions in current use are not generally covariant. In particular, the moving puncture gauge \cite{Bona:1994dr,Alcubierre:2002kk}, commonly 
used with BSSN, is not covariant.  Likewise, the gauge conditions \cite{Pretorius:2006tp,Szilagyi:2009qz} used 
with GH are not covariant.  

For a non--covariant gauge condition, the form of the condition depends on the choice of coordinate system. 
In practice, one writes the gauge  in a form intended for use with ``Cartesian coordinates". This prescription is not  well--defined 
since Cartesian coordinates are not unique. Nevertheless, one chooses some set of initial data expressed in coordinates that are ``topologically Cartesian"  
and then writes the gauge equations in these coordinates.  One can then use the transformation rules for the fields to write the gauge conditions in terms of 
other coordinate systems. As we will show, these calculations can be tedious. 

The alternative discussed in this paper is to rewrite the gauge conditions in a form that is generally covariant under foliation preserving 
transformations. The gauge equations then take the same form in any coordinate system, and there is no need derive a new form 
for each coordinate system that one might consider. To carry out the analysis, we first derive the transformation rules for the ADM variables $\alpha$, $\beta^a$, 
$g_{ab}$, $K_{ab}$, and their time derivatives, under the foliation preserving transformations (\ref{foliationpreservingtransf}). Next, we introduce a 
set of background fields that allows us to group terms into combinations that transform as tensors. The gauge conditions for the BSSN and GH systems are 
then written in manifestly covariant form using these tensor combinations. 

NOTE:
Cite \cite{Brown:2009dd} and say here we have an extension. Likewise cite \cite{Brown:2011qg}, and describe current results as an extension. 
Somewhere reference \cite{Brown:2008cca}, and perhaps Hilditch/Richter papers as well?
%%%%%%%%%%%%%%%%%%%%%%%%%%%%%%%%%%%%%%%%%%
\section{Transformation Rules}\label{transform}
In the 3+1 formalism, spacetime is foliated by spacelike slices of constant coordinate time $t$. The spacetime metric is written as 
\begin{equation}\label{sptmgin3+1form}
	{}^{(4)}g_{\mu\nu} = (-\alpha^2 + \beta^{c}\beta_{c})\delta_\mu^0\delta_\nu^0 
	+ 2\beta_{a}\delta_{(\mu}^a\delta_{\nu)}^0 + g_{ab} \delta^a_\mu \delta^b_\nu \ ,
\end{equation}
where $\alpha$ is the lapse function, $\beta^{a}$ the shift vector and $g_{ab}$ the spatial metric. 
The extrinsic curvature of the spacelike slices is defined by
\begin{equation}\label{extrinsic}
K_{ab} = - \frac{1}{2\alpha}\left[\partial_{t}g_{ab} - 2D_{(a}\beta_{b)}\right] \ .
\end{equation}
Parentheses around the indices $a$ and $b$ denote symmetrization. 

The covariant normal to each slice of the foliation is  
\begin{equation}\label{normal covector}
n_{\mu} = -\alpha\delta^{0}_{\mu} \ ,
\end{equation}
with dual 
\begin{equation}\label{normal vector}
n^{\mu} = (\delta^{\mu}_{0} - \beta^{c}\delta^{\mu}_{c})/\alpha \ .
\end{equation}
The normal satisfies $n_{\mu}n^{\mu} = 1$. Let us define the operator
\begin{equation}\label{projection 1}
X^{\mu}_{a} = \delta^{\mu}_{a}
\end{equation}
that projects a spacetime covector into a spatial covector. The covariant form of this operator is
\begin{equation}\label{projection 2}
X^{a}_{\mu} = \delta^{a}_{\mu} + \beta^{a}\delta^{0}_{\mu} \ .
\end{equation}
Thus, we have $X^{\mu}_{a}X^{b}_{\mu} = \delta^{b}_{a}$ and $X^{\mu}_{a}n_{\mu} = 0$. 

With these definitions, the spacetime metric can be written  as
\begin{equation}\label{spacetime metric 3 + 1}
^{(4)}g_{\mu\nu} = g_{ab}X^{a}_{\mu}X^{b}_{\nu} - n_{\mu}n_{\nu} \ .
\end{equation}
Note that spacetime indices $\mu$, $\nu$, {\em etc.} are raised and lowered with the spacetime metric $^{(4)}g_{\mu\nu}$ and its inverse $^{(4)}g^{\mu\nu}$. 
Spatial indices $a$, $b$, {\em etc.} are raised and lowered with the spatial metric $g_{ab}$ and its inverse $g^{ab}$. 

The foliation preserving transformation (\ref{foliationpreservingtransf}) consists of a time--dependent spatial coordinate transformation and
a time reparameterization. The inverse transformation is  
\begin{subequations}
\begin{eqnarray}
t & = & t(t') \ ,\\
x^{a} & = & x^{a}(t',x^{a'}) \ .
\end{eqnarray}
\end{subequations}
Using the chain rule for partial differentiation, we have
\[
\delta^{a}_{b} = \frac{\partial x^{a}}{\partial x^{b}} = \frac{\partial x^{a}}{\partial x^{\mu'}}\frac{\partial x^{\mu'}}{\partial x^{b}} = \frac{\partial x^{a}}{\partial t'}\frac{\partial t'}{\partial x^{b}} + \frac{\partial x^{a}}{\partial x^{a'}}\frac{\partial x^{a'}}{\partial x^{b}} \ .
\]
Since $\partial t'/\partial x^{b} = 0$, this result reduces to the important identity  
\begin{equation}
\frac{\partial x^{a}}{\partial x^{a'}}\frac{\partial x^{a'}}{\partial x^{b}} = \delta^{a}_{b} \ .
\end{equation}
Likewise, we find
\begin{equation}
\frac{\partial x^{a'}}{\partial x^{a}}\frac{\partial x^{a}}{\partial x^{b'}} = \delta^{a'}_{b'} 
\end{equation}
by differentiating Eq.~(\ref{foliationpreservingtransf}b). 

Because the spacetime metric transforms as a type $0\choose 2$ tensor under spacetime coordinate transformations, we have
\begin{equation}\label{spacetime metric}
^{(4)}g_{\mu'\nu'} =~^{(4)}g_{\mu\nu}\frac{\partial x^{\mu}}{\partial x^{\mu'}}\frac{\partial x^{\nu}}{\partial x^{\nu'}} 
= (g_{ab}X^{a}_{\mu}X^{b}_{\nu} - n_{\mu}n_{\nu})\frac{\partial x^{\mu}}{\partial x^{\mu'}}\frac{\partial x^{\nu}}{\partial x^{\nu'}} \ .
\end{equation}
By setting the indices $\mu'$, $\nu'$ equal to $a'$, $b'$, we obtain 
\begin{eqnarray*}
g_{a'b'} & = & (g_{ab}X^{a}_{\mu}X^{b}_{\nu} - n_{\mu}n_{\nu})\frac{\partial x^{\mu}}{\partial x^{a'}}\frac{\partial x^{\nu}}{\partial x^{b'}}\notag\\
& = & g_{ab}(\frac{\partial x^{a}}{\partial x^{a'}} + \beta^{a}\frac{\partial t}{\partial x^{a'}})(\frac{\partial x^{b}}{\partial x^{b'}} + \beta^{b}\frac{\partial t}{\partial x^{b'}}) - \alpha^2\frac{\partial t}{\partial x^{a'}}\frac{\partial t}{\partial x^{b'}} \ .
\end{eqnarray*}
Since $\partial t/\partial x^{a'} = 0$, the transformation rule for the spatial metric becomes
\begin{equation}\label{transfruleforgab}
g_{a'b'}  = g_{ab}\frac{\partial x^{a}}{\partial x^{a'}}\frac{\partial x^{b}}{\partial x^{b'}} \ .
\end{equation}
The spatial metric transforms as a type $0 \choose 2$ tensor under spatial coordinate transformations, 
and as a scalar under time reparameterization.
The transformation rule for the inverse spatial metric is 
\begin{equation}
g^{a'b'} = g^{ab}\frac{\partial x^{a'}}{\partial x^{a}}\frac{\partial x^{b'}}{\partial x^{b}}\label{spatial metric} \ .
\end{equation}
That is, $g^{ab}$ is a type $2 \choose 0$ tensor under spatial transformations and a scalar under time reparameterization. 
The transformation rule for the determinant of the spatial metric, $g$, is found by taking the determinant of Eq.~(\ref{transfruleforgab}):
\begin{equation}\label{transfrulefordetg}
g' = g\left| \frac{\partial x}{\partial x'} \right|^{2} \ .
\end{equation}
Here, $|\partial x/\partial x'|$ is the Jacobian of the transformation; that is, the determinant of $\partial x^a/\partial x^{a'}$. 
Equation (\ref{transfrulefordetg}) shows that $g$ transforms as a weight +2 density under spatial coordinate transformations, and as a scalar 
under time reparameterization. 
 
Now consider the  shift vector $\beta^{a}$. From Eq.~(\ref{sptmgin3+1form}) with $\mu = a$ and $\nu = 0$, we have
\begin{equation}
{}^{(4)}g_{a0} = \beta_{a}\label{shift} \ .
\end{equation}
Likewise, we can write $\beta_{a'} = {}^{(4)}g_{a'0'}$. 
Then Eq.~(\ref{spacetime metric}) with $\mu' = a'$ and $\nu' = 0'$ gives
\begin{align*}
\beta_{a'} & = (g_{ab}X^{a}_{\mu}X^{b}_{\nu} - n_{\mu}n_{\nu})\frac{\partial x^{\mu}}{\partial x^{a'}}\frac{\partial x^{\nu}}{\partial t'}\\
& = g_{ab}X^{a}_{\mu}X^{b}_{\nu}\delta_{c}^{\mu}\frac{\partial x^{c}}{\partial x^{a'}}\frac{\partial x^{\nu}}{\partial t'}\\
& = g_{ab}(\delta^{b}_{\nu} + \beta^{b}\delta^{0}_{\nu})\frac{\partial x^{a}}{\partial x^{a'}}\frac{\partial x^{\nu}}{\partial t'}\\
& = \beta_{a}\frac{\partial x^{a}}{\partial x^{a'}}\frac{\partial t}{\partial t'} + g_{ab}\frac{\partial x^{a}}{\partial x^{a'}}\frac{\partial x^{b}}{\partial t'} \ ,
\label{shift covector transform}
\end{align*}
which can be combined with Eq.~(\ref{spatial metric}) to yield
\begin{equation*}
\beta^{a'}  = \beta_{b'}g^{a'b'}
= \left(\beta_{b}\frac{\partial x^{b}}{\partial x^{b'}}\frac{\partial t}{\partial t'} 
+ g_{cd}\frac{\partial x^{c}}{\partial x^{b'}}\frac{\partial x^{d}}{\partial t'}\right)g^{ef}\frac{\partial x^{a'}}{\partial x^{e}}\frac{\partial x^{b'}}{\partial x^{f}} \ .
\end{equation*}
This result simplifies to the transformation rule for the shift vector: 
\begin{equation}
\beta^{a'} = \beta^{a}\frac{\partial x^{a'}}{\partial x^{a}}\frac{\partial t}{\partial t'} 
+ \frac{\partial x^{a'}}{\partial x^{a}}\frac{\partial x^{a}}{\partial t'}\label{shift vector transform}
\end{equation}
The shift does not transform as a simple tensor. Nevertheless, we will continue to refer to the shift as a {\em vector} since it 
does transform as a vector under time--independent transformations of the spatial coordinates. 

Now consider the lapse function $\alpha$. Equation (\ref{sptmgin3+1form}) with $\mu = \nu = 0$ gives
\begin{equation}
{}^{(4)}g_{00} = -\alpha^2 + g_{ab}\beta^{a}\beta^{b}\label{lapse} \ ,
\end{equation}
which can be written in terms of primed variables as $\alpha'^2 = g_{a'b'}\beta^{a'}\beta^{b'} - {}^{(4)}g_{0'0'}$. 
Using the results from Eqs.~(\ref{spacetime metric}), (\ref{transfruleforgab}) and (\ref{shift vector transform}), a short calculation gives 
$\alpha'^{2} = \alpha^2 (\partial t/\partial t')^2$. 
Therefore, 
\begin{equation}
\alpha' = \alpha\frac{\partial t}{\partial t'} \ ,\label{lapse transform} 
\end{equation}
and we see that the lapse function transforms as a scalar under spatial coordinate transformations and a weight +1 density under time reparameterization. 
 
We now consider transformation rules for derivatives of the lapse, shift, and spatial metric. Let us briefly review the chain rule. The chain rule in spacetime domain states as below
\begin{equation*}
\partial_{\mu'} = \frac{\partial x^{\mu}}{\partial x^{\mu'}}\partial_{\mu}
\end{equation*}
Therefore, if $\mu' = t'$, we have that
\begin{equation}\label{time derivative}
\partial_{t'} = \frac{\partial x^{c}}{\partial t'}\partial_{c} + \frac{\partial t}{\partial t'}\partial_{t}
\end{equation}
and if $\mu' = a'$, since $\partial t/\partial x^{a'} = 0$, we have
\begin{equation}\label{spatial derivative}
\partial_{a'} = \frac{\partial x^{c}}{\partial x^{a'}}\partial_{c}
\end{equation}
From Eq.~\ref{time derivative} and Eq.~\ref{spatial derivative} we can tell that the spatial derivative transforms just like a type $0 \choose 1$ tensor without introducing any extra terms. On the other hand, in the time derivative transformations, there are one extra term $(\partial t/\partial t')\partial_{t}$ involved, since $\partial t/\partial t' \ne 0$. According to this analysis, the spatial Christoffel symbols transform regularly, since it is only constructed by the spatial metric and its spatial derivatives. 
\begin{equation}
\Gamma^{a'}_{~b'c'} = \Gamma^{a}_{~bc}\frac{\partial x^{a'}}{\partial x^{a}}\frac{\partial x^{b}}{\partial x^{b'}}\frac{\partial x^{c}}{\partial x^{c'}} + \frac{\partial x^{a'}}{\partial x^{d}}\frac{\partial^{2} x^{d}}{\partial x^{b'}\partial x^{c'}}
\end{equation}

Furthermore, the spatial covariant derivative of a tensor also transforms as a tensor. Specifically, the spatial covariant derivative of the shift covector transforms as
\begin{align}
D_{a'}\beta_{b'} &= D_{a'}\left(\beta_{b}\frac{\partial x^{b}}{\partial x^{b'}}\frac{\partial t}{\partial t'}\right) + D_{a'}\left(g_{bc}\frac{\partial x^{b}}{\partial x^{b'}}\frac{\partial x^{c}}{\partial t'}\right)\notag\\
& = \frac{\partial t}{\partial t'}D_{a'}\left(\beta_{b}\frac{\partial x^{b}}{\partial x^{b'}}\right) + D_{a'}\left(g_{b'c'}\frac{\partial x^{c'}}{\partial x^{c}}\frac{\partial x^{c}}{\partial t'}\right)\notag\\
& = D_{a}\beta_{b}\frac{\partial x^{a}}{\partial x^{a'}}\frac{\partial x^{b}}{\partial x^{b'}}\frac{\partial t}{\partial t'} + g_{b'c'}D_{a'}\left(\frac{\partial x^{c'}}{\partial x^{c}}\frac{\partial x^{c}}{\partial t'}\right)
\end{align} 
The spatial partial derivative of $\alpha$, which appears in the advection term of the 1 + log slicing condition, transforms as 
\begin{align}
\partial_{a'}\alpha' & = \partial_{a'}\left(\alpha \frac{\partial t}{\partial t'}\right)\\
& = \frac{\partial t}{\partial t'}\partial_{a'}\alpha\\
& = \frac{\partial t}{\partial t'}\frac{\partial x^{c}}{\partial x^{a'}}\partial_{c}\alpha
\end{align}
It is a type $0 \choose 1$ tensor under spatial diffeomorphism and a weight +1 density under time reparameterization.

Due to the extra term introduced by time derivative transformation, the time derivative of $\alpha$ raises a more interesting case, 
\begin{align}
\partial_{t'}\alpha' & = \partial_{t'}\left(\alpha\frac{\partial t}{\partial t'}\right)\notag\\
& = \frac{\partial t}{\partial t'}\partial_{t'}\alpha + \alpha\frac{\partial^{2}t}{\partial {t'}^{2}}\notag\\
& = \partial_{t}\alpha\left(\frac{\partial t}{\partial t'}\right)^{2} + \partial_{c}\alpha\frac{\partial x^{c}}{\partial t'}\frac{\partial t}{\partial t'} + \alpha\frac{\partial^{2}t}{\partial {t'}^{2}}
\end{align}

Now let's examine the time derivative of the spatial metric
\begin{align}
\partial_{t'}g_{a'b'} & = (\partial_{t'}g_{ab})\frac{\partial x^{a}}{\partial x^{a'}}\frac{\partial x^{b}}{\partial x^{b'}} + g_{ab}\partial_{t'}\left(\frac{\partial x^{a}}{\partial x^{a'}}\frac{\partial x^{b}}{\partial x^{b'}}\right)\notag\\
& =  (\partial_{t}g_{ab})\frac{\partial t}{\partial t'}\frac{\partial x^{a}}{\partial x^{a'}}\frac{\partial x^{b}}{\partial x^{b'}} + (\partial_{c}g_{ab})\frac{\partial x^{c}}{\partial t'}\frac{\partial x^{a}}{\partial x^{a'}}\frac{\partial x^{b}}{\partial x^{b'}} + g_{ab}\partial_{t'}\left(\frac{\partial x^{a}}{\partial x^{a'}}\frac{\partial x^{b}}{\partial x^{b'}}\right)
\end{align}

Finally the time derivative of $\beta^{a}$ transforms as
\begin{align}
\partial_{t'}\beta^{a'} & = \partial_{t'}\left(\beta^{a}\frac{\partial x^{a'}}{\partial x^{a}}\frac{\partial t}{\partial t'} + \frac{\partial x^{a'}}{\partial x^{a}}\frac{\partial x^{a}}{\partial t'}\right)\notag\\
& = \frac{\partial t}{\partial t'}\partial_{t}\left(\beta^{a}\frac{\partial x^{a'}}{\partial x^{a}}\frac{\partial t}{\partial t'} + \frac{\partial x^{a'}}{\partial x^{a}}\frac{\partial x^{a}}{\partial t'}\right) + \frac{\partial x^{c}}{\partial t'}\partial_{c}\left(\beta^{a}\frac{\partial x^{a'}}{\partial x^{a}}\frac{\partial t}{\partial t'} + \frac{\partial x^{a'}}{\partial x^{a}}\frac{\partial x^{a}}{\partial t'}\right)\notag\\
& = \partial_{t}\beta^{a}\frac{\partial x^{a'}}{\partial x^{a}}\left(\frac{\partial t}{\partial t'}\right)^{2} + \beta^{a}\frac{\partial t}{\partial t'}\partial_{t}\left(\frac{\partial x^{a'}}{\partial x^{a}}\frac{\partial t}{\partial t'}\right) + \frac{\partial t}{\partial t'}\partial_{t}\left(\frac{\partial x^{a'}}{\partial x^{a}}\frac{\partial x^{a}}{\partial t'}\right)\notag\\
& + \frac{\partial x^{c}}{\partial t'}\partial_{c}\left(\beta^{a}\frac{\partial x^{a'}}{\partial x^{a}}\frac{\partial t}{\partial t'} + \frac{\partial x^{a'}}{\partial x^{a}}\frac{\partial x^{a}}{\partial t'}\right)
\end{align}

%%%%%%%%%%%%%%%%%%%%%%%%%%%%%%%%%%%%%%%%
\section{Tensorial Variables}\label{tensorial}
In Sec.~\ref{transform}, we listed the transformation rules of all the important 3 + 1 variables and as one can easily notice that some of them do not transform tensorially. In order to solve this, we introduce the backgrond metric denoted as ${\bar g}_{\mu\nu}$. The background metric was initially introduced to make the Christoffel symbols transform as a tensor and it is usually defined as a trivial metric, i.e., ${\bar g}_{\mu\nu} = diag(-1,1,1,1)$. Therefore, we have ${\bar g} = 1, $${\bar \Gamma}^{a}_{~bc} = 0$, ${\bar \alpha} = 1$ and ${\bar \beta}^{a} = 0$. 

The background shift vector ${\bar \beta}^{a}$ follows the same transformation rule of $\beta^{a}$
\[
{\bar \beta}^{a'} = {\bar \beta}^{a}\frac{\partial x^{a'}}{\partial x^{a}}\frac{\partial t}{\partial t'} + \frac{\partial x^{a'}}{\partial x^{a}}\frac{\partial x^{a}}{\partial t'}
\]
and hence the difference between $\beta^{a}$ and ${\bar \beta}^{a}$ transform as a type $1 \choose 0$ tensor under spatial diffeomorphism and a weight +1 density under time reparameterization. 
\begin{equation}
\Delta \beta^{a'} = \Delta \beta^{a}\frac{\partial x^{a'}}{\partial x^{a}}\frac{\partial t}{\partial t'}
\end{equation}
where $\Delta \beta^{a} \equiv \beta^{a} - {\bar \beta}^{a}$.

In addition, the tensorial form of Christoffel symbols is $\Delta \Gamma^{a}_{~bc} \equiv \Gamma^{a}_{~bc} - {\bar \Gamma}^{a}_{~bc}$ and
\begin{equation}
\Delta \Gamma^{a'}_{~b'c'} = \Delta \Gamma^{a}_{~bc}\frac{\partial x^{a'}}{\partial x^{a}}\frac{\partial x^{b}}{\partial x^{b'}}\frac{\partial x^{c}}{\partial x^{c'}}
\end{equation}
As we can tell, it transforms as a type $1 \choose 2$ tensor under spatial diffeomorphism and a scalar under time reparameterization. 

The invariant time derivative of the spatial metric is defined as
\begin{equation}
D_{t}g_{ab} \equiv \partial_{t}g_{ab} - 2D_{(a}\beta_{b)}
\end{equation}
The calculation is tedious but one can always check that $D_{t}g_{ab}$ transforms as a type $0 \choose 2$ tensor under spatial diffeomorphism and a weight +1 density under time reparameterization. 
\begin{equation}
D_{t'}g_{a'b'} = D_{t}g_{ab}\frac{\partial x^{a}}{\partial x^{a'}}\frac{\partial x^{b}}{\partial x^{b'}}\frac{\partial t}{\partial t'}
\end{equation}

As a result, since the extrinsic curvature is defined as
\begin{equation}
K_{ab} = -\frac{1}{2\alpha}[\partial_{t}g_{ab} - 2D_{(a}\beta_{b)}] = -\frac{1}{2\alpha}D_{t}g_{ab}
\end{equation}
it transforms as a type $0 \choose 2$ tensor under the spatial diffeomorphism and a scalar under the time reparameterization
\begin{equation}
K_{a'b'} = K_{ab}\frac{\partial x^{a}}{\partial x^{a'}}\frac{\partial x^{b}}{\partial x^{b'}}
\end{equation}

In order to construct an invariant form of the time derivative of the lapse function $D_{t}\alpha$ such that it transforms as a scalar under spatial diffeomorphism and a weight +2 density under time reparameterization
\[
D_{t'}\alpha' = D_{t}\alpha\left(\frac{\partial t}{\partial t'}\right)^{2}
\]
We define $D_{t}\alpha$ as
\begin{equation}
D_{t}\alpha \equiv \partial_{t}\alpha - \beta^{c}\partial_{c}\alpha - \frac{\alpha}{{\bar \alpha}}(\partial_{t}{\bar \alpha} - {\bar \beta}^{c}\partial_{c}{\bar \alpha})
\end{equation}
and check how it transforms
\begin{align}
D_{t'}\alpha' & = \partial_{t'}\alpha' - \beta^{c'}\partial_{c'}\alpha' - \frac{\alpha'}{{\bar \alpha}'}(\partial_{t'}{\bar \alpha}' - {\bar \beta}^{c'}\partial_{c'}{\bar \alpha}')\notag\\
& = \partial_{t}\alpha\left(\frac{\partial t}{\partial t'}\right)^{2} + \partial_{c}\alpha\frac{\partial x^{c}}{\partial t'}\frac{\partial t}{\partial t'} + \alpha\frac{\partial^{2}t}{\partial {t'}^{2}} - \left(\beta^{c}\frac{\partial x^{c'}}{\partial x^{c}}\frac{\partial t}{\partial t'} + \frac{\partial x^{c'}}{\partial x^{c}}\frac{\partial x^{c}}{\partial t'}\right)\frac{\partial t}{\partial t'}\frac{\partial x^{d}}{\partial x^{c'}}\partial_{d}\alpha\notag\\
& - \frac{\alpha}{{\bar \alpha}}\left[\partial_{t}{\bar \alpha}\left(\frac{\partial t}{\partial t'}\right)^{2} + \partial_{c}{\bar \alpha}\frac{\partial x^{c}}{\partial t'}\frac{\partial t}{\partial t'} + {\bar \alpha}\frac{\partial^{2}t}{\partial {t'}^{2}} - \left({\bar\beta}^{c}\frac{\partial x^{c'}}{\partial x^{c}}\frac{\partial t}{\partial t'} + \frac{\partial x^{c'}}{\partial x^{c}}\frac{\partial x^{c}}{\partial t'}\right)\frac{\partial t}{\partial t'}\frac{\partial x^{d}}{\partial x^{c'}}\partial_{d}{\bar\alpha}\right]\notag\\
& = D_{t}\alpha\left(\frac{\partial t}{\partial t'}\right)^{2}
\end{align}
For formulating the invariant form of the shift vector's time derivative, $\partial_{t}\Delta \beta^{a}$ would be a suitable candidate, since $\Delta \beta^{a}$ itself behaves tensorially already. Let's try to transform $\partial_{t} \Delta \beta^{a}$
\begin{align}\label{time derivative delta beta} 
\partial_{t'}\Delta \beta^{a'} & = \left(\partial_{t'}\Delta \beta^{a}\right)\frac{\partial x^{a'}}{\partial x^{a}}\frac{\partial t}{\partial t'} + \Delta \beta^{a}\partial_{t'}\left(\frac{\partial x^{a'}}{\partial x^{a}}\frac{\partial t}{\partial t'}\right)\notag\\
& = \partial_{t}\Delta \beta^{a}\frac{\partial x^{a'}}{\partial x^{a}}\left(\frac{\partial t}{\partial t'}\right)^{2} + \partial_{c}\Delta \beta^{a}\frac{\partial x^{c}}{\partial t'}\frac{\partial x^{a'}}{\partial x^{a}}\frac{\partial t}{\partial t'}\notag\\
& + \Delta \beta^{a}\frac{\partial x^{a'}}{\partial x^{a}}\frac{\partial^{2}t}{\partial t'^{2}} + \Delta \beta^{a}\frac{\partial^{2} x^{a'}}{\partial x^{c}\partial x^{a}}\frac{\partial x^{c}}{\partial t'}\frac{\partial t}{\partial t'} + \Delta \beta^{a}\frac{\partial^{2}x^{a'}}{\partial t\partial x^{a}}\left(\frac{\partial t}{\partial t'}\right)^{2}
\end{align}
As we expected, there are several extra terms in Eq.~\ref{time derivative delta beta} prevent $\partial_{t}\Delta\beta^{a}$ from being a tensor density. 

By observing the transformation rules of $\partial_{t}\Delta\beta^{a}$, $\beta^{a}$, $D_{a}\beta_{b}$ and $\partial_{a}\alpha$, we propose an ansatz of $D_{t}\beta^{a}$ such that $D_{t}\beta^{a}$ transforms as a type $1 \choose 0$ tensor under spatial diffeomorphism and a weight +2 density under time reparameterization, 
\begin{equation}
D_{t'}\beta^{a'} = D_{t}\beta^{a}\frac{\partial x^{a'}}{\partial x^{a}}\left(\frac{\partial t}{\partial t'}\right)^{2}
\end{equation}
where $D_{t}\beta^{a}$ is defined as
\begin{equation}
D_{t}\beta^{a} \equiv \partial_{t}\Delta \beta^{a} - \frac{\Delta \beta^{a}}{{\bar \alpha}}(\partial_{t}{\bar \alpha} - {\bar \beta}^{a}\partial_{a}{\bar \alpha}) + c_{1}\beta^{b}{\bar D}_{b}\beta^{a} + c_{2}\beta^{b}{\bar D}_{b}{\bar \beta}^{a} + c_{3}{\bar \beta}^{b}{\bar D}_{b}\beta^{a} + c_{4}{\bar \beta}^{b}{\bar D}_{b}{\bar \beta}^{a}
\end{equation}
and $c_{1}$, $c_{2}$, $c_{3}$ and $c_{4}$ are all constants. 

Determining the values of these constants involves tedious calculation, so we skip to the conclusion here. The constants obey the following constraints
\begin{align*}
c_{1} + c_{3} & = -1\\
c_{2} + c_{4} & = 1\\
c_{1} + c_{2} & = 1\\
c_{3} + c_{4} & = -1 
\end{align*}
so that $D_{t} \beta^{a}$ transforms as a tensor density. 

Therefore, we define $\sigma \equiv c_{1}$ and write the rest of those constants in terms of $\sigma$ to get the final formulation of $D_{t}\beta^{a}$ as below
\begin{equation}
D_{t}\beta^{a} \equiv \partial_{t}\Delta \beta^{a} - \frac{\Delta \beta^{a}}{{\bar \alpha}}(\partial_{t}{\bar \alpha} - {\bar \beta}^{a}\partial_{a}{\bar \alpha}) + \sigma\beta^{b}{\bar D}_{b}\beta^{a} + (1-\sigma)\beta^{b}{\bar D}_{b}{\bar \beta}^{a} - (1 + \sigma){\bar \beta}^{b}{\bar D}_{b}\beta^{a} + \sigma{\bar \beta}^{b}{\bar D}_{b}{\bar \beta}^{a}
\end{equation}
In the rest of this paper, we take $\sigma = -1$ since there is one $-\beta^{c}\partial_{c}\beta^{a}$ term in the regular Gamma-driver shift condition. Hence the definition of $D_{t}\beta^{a}$ becomes
\begin{equation}
D_{t}\beta^{a} \equiv \partial_{t}\Delta \beta^{a} - \frac{\Delta \beta^{a}}{{\bar \alpha}}(\partial_{t}{\bar \alpha} - {\bar \beta}^{a}\partial_{a}{\bar \alpha}) - \beta^{b}{\bar D}_{b}\Delta \beta^{a} + \Delta \beta^{b}{\bar D}_{b}{\bar \beta}^{a}
\end{equation}

%%%%%%%%%%%%%%%%%%%%%%%%%%%%%%%%%%%%%%%
\section{Covariant Gauge Condition}\label{gauge}
We studied the transformation rules of all the important variables in 3 + 1 splitting mechanism under the foliation preserving coordinate transformation in Sec.~\ref{transform} and Sec.~\ref{tensorial}, now we will use these results to construct the invariant form of the 1 + log slicing condition and Gamma-driver shift equation. 

The regular 1 + log slicing condition reads as
\begin{equation}\label{1 + log slicing}
\partial_{t}\alpha - \beta^{c}\partial_{c}\alpha = -2\alpha K
\end{equation}
Due to the analysis in Sec.~\ref{transform}, one can notice that neither the right hand side nor the left hand side transforms tensorially. We start by replacing the left hand side by $D_{t}\alpha$. Since $D_{t'}\alpha' = D_{t}\alpha(\partial t/\partial t')^{2}$, in order to balance the right hand side, we modify it to be $-2\alpha {\bar \alpha}K$. After this modification, we have
\begin{align*}
D_{t'}\alpha' & = D_{t}\alpha\left(\frac{\partial t}{\partial t'}\right)^{2}\\
& = -2\alpha{\bar \alpha}K\left(\frac{\partial t}{\partial t'}\right)^{2}\\
& = -2\left(\alpha\frac{\partial t}{\partial t'}\right)\left({\bar \alpha}\frac{\partial t}{\partial t'}\right)K\\
& = -2\alpha'{\bar \alpha}'K'
\end{align*}
Therefore, we have reached the invariant form of 1 + log slicing condition as below
\begin{equation}\label{invariant 1 + log slicing}
D_{t}\alpha = -2\alpha{\bar \alpha}K
\end{equation}

The Gamma-driver shift condition is usually written as a system of two first order equations
\begin{subequations}\label{Gamma-driver shift set}
\begin{align}
\partial_{t}\beta^{a} & = \beta^{b}\partial_{b}\beta^{a} + \frac{3}{4}B^{a}\\
\partial_{t}B^{a} & = \beta^{b}\partial_{b}B^{a} + \partial_{t}{\tilde \Gamma}^{a} - \beta^{b}\partial_{b}{\tilde \Gamma}^{a} - \eta B^{a}
\end{align}
\end{subequations}
where ${\tilde \Gamma}^{a}$ is the trace of the conformal Christoffel symbols defined as
\begin{equation}
{\tilde \Gamma}^{a} \equiv \sqrt{g}^{2/3}\left(\Gamma^{a}_{~bc}g^{bc} + \frac{1}{3}g^{ab}\Gamma^{c}_{~bc}\right)
\end{equation}
and $\eta$ is a constant parameter. 

We can follow the same strategy to alter the Gamma-driver shift condition, replacing $\partial_{t}\beta^{a}$ by $D_{t}\beta^{a}$, changing every non-tensorial term to its corresponding tensorial term and balance the transformation factor on both sides. Therefore, we have
\begin{subequations}\label{invariant Gamma-driver shift set}
\begin{align}
\partial_{t}\Delta \beta^{a} & = \beta^{b}{\bar D}_{b}\Delta\beta^{a} - \Delta \beta^{b}{\bar D}_{b}{\bar \beta}^{a} + \frac{\Delta\beta^{a}}{{\bar \alpha}}(\partial_{t}{\bar \alpha} - {\bar \beta}^{b}\partial_{b}{\bar \alpha}) + \frac{3}{4}{\bar \alpha}B^{a}\\
\partial_{t}B^{a} & = \partial_{t}\Delta {\tilde \Gamma}^{a} + \beta^{b}{\bar D}_{b}(B^{a} - \Delta{\tilde \Gamma}^{a}) - (B^{b} - \Delta{\tilde \Gamma}^{b}){\bar D}_{b}{\bar \beta}^{a} + \frac{B^{a} - \Delta{\tilde \Gamma}^{a}}{{\bar \alpha}}(\partial_{t}{\bar \alpha} - {\bar \beta}^{b}\partial_{b}{\bar \alpha}) - \eta{\bar \alpha}B^{a}
\end{align}
\end{subequations}
where
\begin{equation}
\Delta {\tilde \Gamma}^{a} \equiv {\bar \alpha}\sqrt{\frac{g}{{\bar g}}}^{2/3}\left(\Delta \Gamma^{a}_{~bc}g^{bc} + \frac{1}{3}g^{ab}\Delta \Gamma^{c}_{~bc}\right)
\end{equation}
In some literatures \cite{vanMeter:2006vi}, the Eqs.~\ref{Gamma-driver shift set} can be integrated with the initial condition $\beta^{a}(t=0) = B^{a}(t=0) = {\tilde \Gamma}^{a}(t = 0) = 0$ and yields a single first order equation of Gamma-driver shift condition
\begin{equation}\label{Gamma-driver shift}
\partial_{t}\beta^{a} = \beta^{b}\partial_{b}\beta^{a} + \frac{3}{4}{\tilde \Gamma}^{a} - \eta \beta^{a}
\end{equation} 
Similarly, we can also integrate the Eqs.~\ref{invariant Gamma-driver shift set} with the same initial condition mentioned before to get the invariant form of Eq.~\ref{Gamma-driver shift}
\begin{equation}\label{invariant Gamma-driver shift}
D_{t}\beta^{a} = \frac{3}{4}{\bar \alpha}\Delta {\tilde \Gamma}^{a} - \eta {\bar \alpha}\Delta\beta^{a}
\end{equation}
Due to its simplicity, we focus our discussion on Eq.~\ref{invariant Gamma-driver shift} instead of Eqs.~\ref{invariant Gamma-driver shift set} in the rest of this paper.

Writing down these invariant gauge conditions creates more possibilities for numerical simulation. When the need of evolving the system in a different coordinates system rises, one can just use the following equations as gauge conditions
\begin{align*}
D_{t'}\alpha' & = -2\alpha'{\bar \alpha}'K'\\
D_{t'}\beta^{a'} & = \frac{3}{4}{\bar \alpha}'\Delta {\tilde \Gamma}^{a'} - \eta {\bar \alpha}'\Delta\beta^{a'}
\end{align*}
while the transformed background metric terms are associated with the original terms as below
\begin{align}
{\bar g}_{a'b'} & = {\bar g}_{ab}\frac{\partial x^{a}}{\partial x^{a'}}\frac{\partial x^{b}}{\partial x^{b'}}\\
{\bar g'} & = {\bar g}|\frac{\partial x}{\partial x'}|^{2}\\
{\bar \alpha}' & = {\bar \alpha}\frac{\partial t}{\partial t'}\\
{\bar \beta}^{a'} & = {\bar \beta}^{a}\frac{\partial x^{a'}}{\partial x^{a}}\frac{\partial t}{\partial t'} + \frac{\partial x^{a'}}{\partial x^{c}}\frac{\partial x^{c}}{\partial t'}\label{background shift}\\
{\bar \Gamma}^{a'}_{~b'c'} & = {\bar \Gamma}^{a}_{~bc}\frac{\partial x^{a'}}{\partial x^{a}}\frac{\partial x^{b}}{\partial x^{b'}}\frac{\partial x^{c}}{\partial x^{c'}} + \frac{\partial x^{a'}}{\partial x^{d}}\frac{\partial x^{d}}{\partial x^{b'}\partial x^{c'}}
\end{align}

We can also achieve the transformed gauge conditions from the regular gauge conditions. All we need to do is using the transformation rules we derived above to write the non-transformed variables in terms of the transformed terms. And we argue as following that these two methods should yield identical transformed gauge conditions. 

Assume we have some variables $v$. Under a change of coordinates, they transform to $v' = T(v)$ where $T$ is some function. We also introduce a 
set of background fields that transform in the same way; thus $\bar v' = T(\bar v)$. The gauge conditions in covariant form are $F^A(v,\bar v) = 0$, 
where the index $A$ ranges over the number of conditions. Since the gauge conditions are covariant, we have 
\begin{equation}
	F^A(v',\bar v') = \xi^A_B F^B(v,\bar v)
\end{equation}
where $\xi^A_B$ is an invertible matrix. That is, 
\begin{equation}\label{invariant equivalence}
	 F^A(v,\bar v) = 0 \Longleftrightarrow F^A(v',\bar v') = 0
\end{equation}
In the original coordinate system, the gauge condition is $F^A(v,\bar v) = 0$ and in the transformed coordinate system, the 
gauge condition is $F^A(v',\bar v') = 0$.

Now we want to consider a particular background. Assume in the original coordinate system, the background fields take the values $\bar v = \bar c$. The gauge 
conditions in original coordinates are $F^A(v,\bar c) = 0$. (This is the non--covariant version of the gauge conditions.) As we discussed above, we have two ways to achieve the transformed gauge conditions. One way: 
\begin{itemize}
	\item Start with the gauge written in covariant form, $F^A(v,\bar v) = 0$.
	\item Change to transformed coordinates, where the gauge is $F^A(v',\bar v') = 0$.
	\item Set the background to $\bar v = \bar c$. But $\bar v' = T(\bar v)$ so that $\bar v' = T(\bar c)$. 
	\item Then the gauge conditions in transformed coordinates read $F^A(v',T(\bar c)) = 0$. 
\end{itemize}
Another way: 
\begin{itemize}
	\item Start with the gauge written in covariant form, $F^A(v,\bar v) = 0$.
	\item Set the background to $\bar v = \bar c$, so the gauge takes the non--covariant form $F^A(v,\bar c) = 0$. 
	\item Now use the transformation rules to replace the original fields with transformed fields; that is, use $v = T^{-1}(v')$. 
	The gauge becomes $F^A(T^{-1}(v'),\bar c) = 0$. 
\end{itemize}
These two results are equivalent. That is, 
\begin{equation}
	F^A(T^{-1}(v'),\bar c) = 0 \Longleftrightarrow F^A(v',T(\bar c)) = 0
\end{equation}
This is guaranteed by the Eq.~\ref{invariant equivalence} above. In Eq.~\ref{invariant equivalence}, just let $v = T^{-1}(v')$ and $\bar v = \bar c$, which implies $\bar v' = T(\bar c)$.
 
We will explain how this scheme works in detail with a specific example in the next section. 

\section{Time Dependent Spatial Rotation Coordinate Transformation}\label{example}
In this section, we examine a specific kind of coordinate transformation, the time dependent spatial rotation coordinate transformation, and demonstrate how the invariant gauge conditions work in action. 

In this coordinate transformation, we keep the $t$ coordinate intact while let the spatial coordinate rotates around the $z$ axis at a constant angular speed $\omega$. Hence the transformed coordinates can be expressed in terms of the original terms as below
\begin{align}
t' & = t\notag\\
x' & = x \cos(\omega t) + y\sin(\omega t)\notag\\
y' & = -x\sin(\omega t) + y\cos(\omega t)\notag\\
z' & = z
\end{align}
and equivalently, 
\begin{align}
t & = t'\notag\\
x & = x'\cos(\omega t') - y'\sin(\omega t')\notag\\
y & = x'\sin(\omega t') + y'\cos(\omega t')\notag\\
z & = z'
\end{align}
We also assume that the background metric is trivial
\begin{align}
{\bar g}_{ab} & = diag(1,1,1)\notag\\
{\bar \alpha} & = 1\notag\\
{\bar \beta} & = 0 
\end{align}
Under this transformation, we have $\partial t/\partial t' = 1$, therefore ${\bar \alpha}' = {\bar \alpha} = 1$ and hence the invariant time derivative of $\alpha$ will collapse to
\begin{equation}
D_{t}\alpha = \partial_{t}\alpha - \beta^{a}\partial_{a}\alpha
\end{equation}
Then the invariant 1 + log slicing condition under this transformation becomes
\begin{equation}\label{transformed 1 + log slicing}
\partial_{t'}\alpha' - \beta^{a'}\partial_{a'}\alpha' = -2\alpha'K'
\end{equation}
One would argue that this equation is identical to the regular 1 + log slicing equation. This is only true under this specific coordinate transformation, especially when we have $\partial t/\partial t' = 1$. One will see the difference in the Gamma-driver shift condition discussed below. 

According to Eq.~\ref{background shift} and the choice that ${\bar \beta}^{a} = 0$, we have that
\[
{\bar \beta}^{a'} = \frac{\partial x^{a'}}{\partial x^{a}}\frac{\partial x^{a}}{\partial t'}
\]
Therefore, for each component, we have
\begin{align*}
{\bar \beta}^{1'} & = -\omega y'\\
{\bar \beta}^{2'} & = \omega x'\\
{\bar \beta}^{3'} & = 0
\end{align*}
Similarly, for the background Christoffel symbols, we have
\[
{\bar \Gamma}^{a'}_{b'c'} = \frac{\partial x^{a'}}{\partial x^{d}}\frac{\partial x^{d}}{\partial x^{b'}\partial x^{c'}}
\]
however, since in this coordinate transformation, $x^{a}$ is a first order function of $x^{a'}$, ${\bar \Gamma}^{a'}_{b'c'}$ vanishes. 

Also, it is easy to check that $|\partial x/\partial x'| = 1$, this result yields that ${\bar g}' = {\bar g} = 1$. According to these results above, the transformed Gamma-driver shift condition becomes
\begin{equation}
\partial_{t'}\beta^{a'} - \beta^{b'}\partial_{b'}\beta^{a'} + 2\beta^{b'}\partial_{b'}{\bar \beta}^{a'} - {\bar \beta}^{b'}\partial_{b'}{\bar \beta}^{a'} = \frac{3}{4}{\tilde \Gamma}^{a'} - \eta \Delta \beta^{a'}
\end{equation}
and for each component, we have
\begin{align}
\partial_{t'}\beta^{1'} - \beta^{b'}\partial_{b'}\beta^{1'} - 2\omega\beta^{2'} + \omega^{2}x' &= \frac{3}{4}{\tilde \Gamma}^{1'} - \eta (\beta^{1'} + \omega y')\label{transformed Gamma-driver shift x}\\
\partial_{t'}\beta^{2'} - \beta^{b'}\partial_{b'}\beta^{2'} + 2\omega\beta^{1'} + \omega^{2}y' &= \frac{3}{4}{\tilde \Gamma}^{2'} - \eta (\beta^{2'} - \omega x')\label{transformed Gamma-driver shift y}\\
\partial_{t'}\beta^{3'} - \beta^{b'}\partial_{b'}\beta^{3'} & = \frac{3}{4}{\tilde \Gamma}^{3'} - \eta \beta^{3'}\label{transformed Gamma-driver shift z}
\end{align}
To verify the transformed gauge conditions(Eq.~\ref{transformed 1 + log slicing} and equations [\ref{transformed Gamma-driver shift x}-\ref{transformed Gamma-driver shift z}]) obtained by the invariant gauge conditions, we try to write down the transformed gauge conditions by transforming the variables in the regular gauge conditions explicitly and compare with the equations we already have. 

According to the transform rules we obtained in Sec.~\ref{transform}, let's write the untransformed variables in terms of the transformed variables. 
\begin{align}
\alpha & = \alpha'\frac{\partial t'}{\partial t} = \alpha'\label{alpha to alpha prime}\\
\beta^{a} & = \left(\beta^{a'} - \frac{\partial x^{a'}}{\partial x^{b}}\frac{\partial x^{b}}{\partial t'}\right)\frac{\partial x^{a}}{\partial x^{a'}}\frac{\partial t'}{\partial t} =   \left(\beta^{a'} - \frac{\partial x^{a'}}{\partial x^{b}}\frac{\partial x^{b}}{\partial t'}\right)\frac{\partial x^{a}}{\partial x^{a'}}\label{beta to beta prime}\\
g_{ab} & = g_{a'b'}\frac{\partial x^{a'}}{\partial x^{a}}\frac{\partial x^{b'}}{\partial x^{b}}\label{metric to metric prime}\\
g & = g'\left(\frac{\partial x'}{\partial x}\right)^{2} = g'\label{det to det prime}\\
\Gamma^{a}_{~bc} & = \left(\Gamma^{a'}_{~b'c'} - \frac{\partial x^{a'}}{\partial x^{d}}\frac{\partial x^{d}}{\partial x^{b'}\partial x^{c'}}\right)\frac{\partial x^{a}}{\partial x^{a'}}\frac{\partial x^{b'}}{\partial x^{b}}\frac{\partial x^{c'}}{\partial x^{c}} = \Gamma^{a'}_{~b'c'}\frac{\partial x^{a}}{\partial x^{a'}}\frac{\partial x^{b'}}{\partial x^{b}}\frac{\partial x^{c'}}{\partial x^{c}}\label{gamma to gamma prime}\\
K & = K'\label{K to K prime}\\
\end{align}
Plug Eq.~\ref{alpha to alpha prime}, Eq.~\ref{beta to beta prime} and Eq.~\ref{K to K prime} into Eq.~\ref{1 + log slicing}, we have
\begin{align}
\partial_{t}\alpha' - \left(\beta^{c'} - \frac{\partial x^{c'}}{\partial x^{d}}\frac{\partial x^{d}}{\partial t'}\right)\frac{\partial x^{c}}{\partial x^{c'}}\partial_{c}\alpha' & = -2\alpha' K'\notag\\
\partial_{t'}\alpha' + \frac{\partial x^{c'}}{\partial t}\partial_{c'}\alpha' - \beta^{c'}\partial_{c'}\alpha' + \frac{\partial x^{c'}}{\partial x^{d}}\frac{\partial x^{d}}{\partial t'}\partial_{c'}\alpha' & = -2\alpha' K'\notag\\
\partial_{t'}\alpha' + \frac{\partial x^{c'}}{\partial t}\partial_{c'}\alpha' - \beta^{c'}\partial_{c'}\alpha' - \frac{\partial x^{c'}}{\partial t}\frac{\partial t}{\partial t'}\partial_{c'}\alpha' & = -2\alpha' K'\notag\\
\partial_{t'}\alpha'  - \beta^{c'}\partial_{c'}\alpha' & = -2\alpha' K'
\end{align}
Hereby we reached the identical equation as Eq.~\ref{transformed 1 + log slicing}.

Now let's plug equations [\ref{beta to beta prime} - \ref{gamma to gamma prime}] into Eq.~\ref{Gamma-driver shift} and write the regular Gamma-driver shift condition in terms of the transformed variables. 

Since this is a long equation, let's first examine the left hand side, 
\begin{align}
& \partial_{t}\left[\left(\beta^{a'} - \frac{\partial x^{a'}}{\partial x^{c}}\frac{\partial x^{c}}{\partial t'}\right)\frac{\partial x^{a}}{\partial x^{a'}}\right] - \left[\left(\beta^{b'} - \frac{\partial x^{b'}}{\partial x^{c}}\frac{\partial x^{c}}{\partial t'}\right)\frac{\partial x^{b}}{\partial x^{b'}}\right]\partial_{b}\left[\left(\beta^{a'} - \frac{\partial x^{a'}}{\partial x^{c}}\frac{\partial x^{c}}{\partial t'}\right)\frac{\partial x^{a}}{\partial x^{a'}}\right]\notag\\
= & \partial_{t}\left(\beta^{a'}\frac{\partial x^{a}}{\partial x^{a'}} - \frac{\partial x^{a}}{\partial t'}\right) - \left(\beta^{b'} - \frac{\partial x^{b'}}{\partial x^{c}}\frac{\partial x^{c}}{\partial t'}\right)\partial_{b'}\left(\beta^{a'}\frac{\partial x^{a}}{\partial x^{a'}} - \frac{\partial x^{a}}{\partial t'}\right)\notag\\
= & \left(\partial_{t'}\beta^{a'}\right)\frac{\partial x^{a}}{\partial x^{a'}} - \left(\beta^{b'}\partial_{b'}\beta^{a'}\right)\frac{\partial x^{a}}{\partial x^{a'}} + 2\beta^{a'}\left(\partial_{t'}\frac{\partial x^{a}}{\partial x^{a'}}\right) - \frac{\partial^{2}x^{a}}{\partial t'^{2}}\notag\\
= & \left(\partial_{t'}\beta^{a'} - \beta^{b'}\partial_{b'}\beta^{a'} - 2\beta^{b'}\frac{\partial x^{c}}{\partial x^{b'}}\frac{\partial^{2}x^{a'}}{\partial x^{c}\partial t} + \frac{\partial^{2} x^{a'}}{\partial t^{2}} + 2\frac{\partial^{2} x^{a'}}{\partial t\partial x^{c}}\frac{\partial x^{c}}{\partial t'}\right)\frac{\partial x^{a}}{\partial x^{a'}}
\end{align}
Then the right hand side is
\begin{align}
\frac{3}{4}\sqrt{g'}^{2/3}\left(\Gamma^{a'}_{~b'c'}g^{b'c'} + \frac{1}{3}g^{a'b'}\Gamma^{c'}_{~b'c'}\right)\frac{\partial x^{a}}{\partial x^{a'}} - \eta\left(\beta^{a'} - \frac{\partial x^{a'}}{\partial x^{b}}\frac{\partial x^{b}}{\partial t'}\right)\frac{\partial x^{a}}{\partial x^{a'}}
\end{align}
Multiply $\partial x^{a'}/\partial x^{a}$ on both sides of the equation, we have
\begin{equation}
\partial_{t'}\beta^{a'} - \beta^{b'}\partial_{b'}\beta^{a'} - 2\beta^{b'}\frac{\partial x^{c}}{\partial x^{b'}}\frac{\partial^{2}x^{a'}}{\partial x^{c}\partial t} + \frac{\partial^{2} x^{a'}}{\partial t^{2}} +  2\frac{\partial^{2} x^{a'}}{\partial t\partial x^{c}}\frac{\partial x^{c}}{\partial t'} = \frac{3}{4}{\tilde \Gamma}^{a'} - \eta\left(\beta^{a'} - \frac{\partial x^{a'}}{\partial x^{b}}\frac{\partial x^{b}}{\partial t'}\right)\label{inverted Gamma-driver shift}
\end{equation}
and one can easily check that Eq.~\ref{inverted Gamma-driver shift} is equivalent to equations [\ref{transformed Gamma-driver shift x} - \ref{transformed Gamma-driver shift z}].
\section{Conclusions}
In this paper, we have investigated the transformation rules of 3 + 1 variables under foliation preserving coordinate transformation. The non-tensor terms are modified to their covariant counterparts so that we are able to formulate the covariant gauge conditions. We have presented the covariant 1 + log slicing and Gamma-driver shift conditions. In the last part of this paper, a specific coordinate transformation example is given as a guideline for people want to implement these invariant gauge conditions in their numerical work. In the future, we plan to use the results in this paper to achieve a symmetric hyperbolic generalized harmonic formulation for general relativity.  

\bibliographystyle{plain}
\bibliography{references}
\end{document}
