\documentclass{article}

\usepackage{amsmath}
\usepackage{amsthm}
\usepackage{graphicx}
\usepackage{float}
\usepackage{listings}
\usepackage{pdfpages}
\usepackage{courier}
\usepackage{color}
\usepackage{mathrsfs}

\begin{document}
\title{Equivalent Gauge Condition}
\author{Meng Cao}
\maketitle

\section{1 + log slicing}
The regular 1 + log slicing equation is
\begin{equation}\label{regular slicing}
{\dot \alpha} = -2\alpha K
\end{equation}
and our invariant form is
\begin{equation}\label{invariant slicing}
\mathring{\alpha} = -2\alpha{\bar \alpha}K
\end{equation}
where
\[
\mathring{\alpha} = ({\dot \alpha} - {\bar \beta}^{a}\partial_{a}\alpha) - \frac{\alpha}{{\bar \alpha}}({\dot {\bar \alpha}} - {\bar \beta}^{a}\partial_{a}{\bar \alpha})
\]
{\color{red} note that this is a little bit different from our previous definition}
{\color{red}
\[
\mathring{\alpha} = ({\dot \alpha} - \beta^{a}\partial_{a}\alpha) - \frac{\alpha}{{\bar \alpha}}({\dot {\bar \alpha}} - {\bar \beta}^{a}\partial_{a}{\bar \alpha})
\]
}
for the transform law, we have that
\begin{align*}
\alpha' & = \alpha\frac{\partial t}{\partial t'}\\
{\bar \alpha}' & = {\bar \alpha}\frac{\partial t}{\partial t'}\\
K' & = K\\
\mathring{\alpha}' & = \mathring{\alpha}\left(\frac{\partial t}{\partial t'}\right)^{2}\\
\end{align*}
We try to break $\mathring{\alpha}$ into two parts
\[
\mathring{\alpha} = {\dot \alpha} + A
\]
and assume they transform separately as
\begin{align*}
{\dot \alpha}' & = ({\dot \alpha} + \delta)\left(\frac{\partial t}{\partial t'}\right)^{2}\\
A' & = (A - \delta)\left(\frac{\partial t}{\partial t'}\right)
\end{align*}
where $\delta$ is the extra term we get when we try to transform ${\dot \alpha}$. In such way, we can recover the transform law of $\mathring{\alpha}$.\\
\\
Therefore, when we try to invert ${\dot \alpha}$ to ${\dot \alpha}'$, we have
\[
{\dot \alpha} = {\dot \alpha}'\left(\frac{\partial t'}{\partial t}\right)^{2} - \delta
\]
Plug this and the transforms laws listed above into Eq.~\ref{regular slicing}, we can have the 1 + log slicing equation in the transformed coordinate system. 
\begin{equation}\label{transformed regular slicing}
{\dot \alpha}'\left(\frac{\partial t'}{\partial t}\right)^{2} - \delta = - 2 \alpha' \frac{\partial t'}{\partial t}K'
\end{equation}
The other way to get the transformed 1 + log slicing equation is from our invariant 1 + log slicing equation. Since it is invariant, we should have
\[
{\dot \alpha}' + A' = - 2\alpha'{\bar \alpha}'K'
\]
Try to rewrite all the background metric terms into unprimed version, we have
\[
{\dot \alpha}' + (A - \delta)\left(\frac{\partial t}{\partial t'}\right)^{2} = -2\alpha' {\bar \alpha}\frac{\partial t}{\partial t'}K'
\]
By assuming the background metric is trivial, we should have ${\bar \alpha} = 1$ and $A = 0$. Therefore, the equation ends up to be
\begin{equation}\label{transformed invariant slicing}
{\dot \alpha}'  - \delta\left(\frac{\partial t}{\partial t'}\right)^{2} = -2\alpha'\frac{\partial t}{\partial t'}K'
\end{equation}
One can easily tell that Eq.~\ref{transformed regular slicing} and Eq.~\ref{transformed invariant slicing} are identical except for a factor of $\left(\frac{\partial t'}{\partial t}\right)^{2}$.
\section{Gamma-driver Shift Equation}
The regular Gamma-driver shift equation is
\begin{equation}\label{regular gamma-driver}
{\dot \beta}^{a} - \beta^{b}\partial_{b}\beta^{a} = \frac{3}{4}\sqrt{g}^{2/3}(\Gamma^{a}_{~bc}g^{bc} + \frac{1}{3}g^{ab}\Gamma^{c}_{~bc}) - \eta \beta^{a} 
\end{equation}
and our invariant form is
\begin{equation}\label{invariant gamma-driver}
\mathring{\beta}^{a} = \frac{3}{4}\sqrt{\frac{g}{{\bar g}}}^{2/3}{\bar \alpha}^{2}(\Delta \Gamma^{a}_{~bc}g^{bc} + \frac{1}{3}g^{ab}\Delta\Gamma^{c}_{~bc}) - \eta {\bar \alpha}\Delta\beta^{a} 
\end{equation}
where
\[
\mathring{\beta}^{a} =  \Delta {\dot \beta}^{a} -\frac{\Delta \beta^{a}}{{\bar \alpha}}({\dot {\bar \alpha}} - {\bar \beta}^{a}\partial_{a}{\bar \alpha}) - \beta^{b}{\bar D}_{b}\beta^{a} + 2\beta^{b}{\bar D}_{b}{\bar \beta}^{a} - {\bar \beta}^{b}{\bar D}_{b}{\bar \beta}^{a}
\]
We can also try to break up $\mathring{\beta}^{a}$ into two parts
\[
\mathring{\beta}^{a} = {\dot \beta}^{a} - \beta^{b}\partial_{b}\beta^{a} + B^{a}
\]
For the transform laws, we have
\begin{align*}
\beta^{a'} & = \beta^{a}\frac{\partial x^{a'}}{\partial x^{a}}\frac{\partial t}{\partial t'} + \frac{\partial x^{a'}}{\partial x^{a}}\frac{\partial x^{a}}{\partial t'}\\
{\bar \beta}^{a'} & = {\bar\beta}^{a}\frac{\partial x^{a'}}{\partial x^{a}}\frac{\partial t}{\partial t'} + \frac{\partial x^{a'}}{\partial x^{a}}\frac{\partial x^{a}}{\partial t'}\\
{\dot \beta}^{a'} - \beta^{b'}\partial_{b'}\beta^{a'} & = ({\dot \beta}^{a} - \beta^{b}\partial_{b}\beta^{a} + \sigma^{a})\frac{\partial x^{a'}}{\partial x^{a}}\left(\frac{\partial t}{\partial t'}\right)^{2}\\
B^{a'} & = (B^{a} - \sigma^{a})\frac{\partial x^{a'}}{\partial x^{a}}\left(\frac{\partial t}{\partial t'}\right)^{2}\\
\mathring{\beta}^{a'} & = \mathring{\beta}^{a}\frac{\partial x^{a'}}{\partial x^{a}}\left(\frac{\partial t}{\partial t'}\right)^{2}\\
\Gamma^{a'}_{~b'c'} & = \Gamma^{a}_{~bc}\frac{\partial x^{a'}}{\partial x^{a}}\frac{\partial x^{b}}{\partial x^{b'}}\frac{\partial x^{c}}{\partial x^{c'}} + \frac{\partial x^{a'}}{\partial x^{a}}\frac{\partial^{2}x^{a}}{\partial x^{b'}\partial x^{c'}}\\
{\bar\Gamma}^{a'}_{~b'c'} & = {\bar \Gamma}^{a}_{~bc}\frac{\partial x^{a'}}{\partial x^{a}}\frac{\partial x^{b}}{\partial x^{b'}}\frac{\partial x^{c}}{\partial x^{c'}} + \frac{\partial x^{a'}}{\partial x^{a}}\frac{\partial^{2}x^{a}}{\partial x^{b'}\partial x^{c'}}\\
g_{a'b'} & = g_{ab}\frac{\partial x^{a}}{\partial x^{a'}}\frac{\partial x^{b}}{\partial x^{b'}}\\
g' & = g\left|\frac{\partial x}{\partial x'}\right|^{2}
\end{align*}
where $\sigma^{a}$ is also the extra term under this transformation.\\
\\ 
If we try to invert $\beta^{a} - \beta^{b}\partial_{b}\beta^{a}$ to $\beta^{a'} - \beta^{b'}\partial_{b'}\beta^{a'}$, we have
\[
\beta^{a} - \beta^{b}\partial_{b}\beta^{a} = (\beta^{a'} - \beta^{b'}\partial_{b'}\beta^{a'})\frac{\partial x^{a}}{\partial x^{a'}}\left(\frac{\partial t'}{\partial t}\right)^{2} - \sigma^{a}
\]
Plug this and the transforms laws listed above into Eq.~\ref{regular gamma-driver}, we can have the Gamma-driver shift equation in the transformed coordinate system. 
\begin{align}\label{transformed gamma-driver}
(\beta^{a'} - \beta^{b'}\partial_{b'}\beta^{a'})\frac{\partial x^{a}}{\partial x^{a'}}\left(\frac{\partial t'}{\partial t}\right)^{2} - \sigma^{a} & = \frac{3}{4}\sqrt{g'}^{2/3}\left|\frac{\partial x'}{\partial x}\right|^{2/3}(\Gamma^{a'}_{~b'c'}g^{b'c'}\frac{\partial x^{a}}{\partial x^{a'}} - g^{b'c'}\frac{\partial^{2} x^{a}}{\partial x^{b'}\partial x^{c'}}\notag\\
& + \frac{1}{3}\Gamma^{c'}_{~b'c'}g^{a'b'}\frac{\partial x^{a}}{\partial x^{a'}} - \frac{1}{3}\frac{\partial x^{c'}}{\partial x^{c}}\frac{\partial^{2} x^{c}}{\partial x^{c'}\partial x^{b'}}\frac{\partial x^{a}}{\partial x^{a'}}g^{a'b'})\notag\\
& - \eta \beta^{a'}\frac{\partial x^{a}}{\partial x^{a'}}\frac{\partial t'}{\partial t} + \eta \frac{\partial x^{a}}{\partial t'}\frac{\partial t'}{\partial t}
\end{align}
The other way to get the transformed Gamma-driver shift equation is from our invariant Gamma-driver shift equation. Since it is invariant, we should have
\[
{\dot \beta}^{a'} - \beta^{b'}\partial_{b'}\beta^{a'} + B^{a'} = \frac{3}{4}\sqrt{\frac{g'}{{\bar g}'}}^{2/3}{\bar \alpha}^{2}(\Delta \Gamma^{a}_{~b'c'}g^{b'c'} + \frac{1}{3}g^{a'b'}\Delta\Gamma^{c'}_{~b'c'}) - \eta {\bar \alpha}'\Delta\beta^{a'}
\]
Try to rewrite all the background metric terms into unprimed version, we have
\begin{align*}
&{\dot \beta}^{a'} - \beta^{b'}\partial_{b'}\beta^{a'} + (B^{a} - \sigma^{a})\frac{\partial x^{a'}}{\partial x^{a}}\left(\frac{\partial t}{\partial t'}\right)^{2} = \notag\\
& \frac{3}{4}\sqrt{\frac{g'}{{\bar g}}}^{2/3}\left|\frac{\partial x'}{\partial x}\right|^{2/3}{\bar \alpha}^{2}\left(\frac{\partial t}{\partial t'}\right)^{2}(\Gamma^{a'}_{~b'c'}g^{b'c'} - {\bar \Gamma}^{a}_{~bc}g^{bc}\frac{\partial x^{a'}}{\partial x^{a}} - \frac{\partial x^{a'}}{\partial x^{a}}\frac{\partial^{2}x^{a}}{\partial x^{b'}\partial x^{c'}}g^{b'c'} )\\
&\frac{1}{3}\Gamma^{c'}_{~b'c'}g^{a'b'} - \frac{1}{3}{\bar\Gamma}^{c}_{~bc}g^{ab}\frac{\partial x^{a'}}{\partial x^{a}} - \frac{1}{3}\frac{\partial x^{c'}}{\partial x^{c}}\frac{\partial^{2}x^{c}}{\partial x^{c'}\partial x^{b'}}g^{a'b'})\\
& - \eta {\bar\alpha}\frac{\partial t}{\partial t'}\beta^{a'} + \eta {\bar\alpha}\frac{\partial t}{\partial t'}{\bar\beta}^{a}\frac{\partial x^{a'}}{\partial x^{a}}\frac{\partial t}{\partial t'} + \eta {\bar \alpha} \frac{\partial t}{\partial t'}\frac{\partial x^{a'}}{\partial x^{a}}\frac{\partial x^{a}}{\partial t'}
\end{align*}
By assuming the background metric is trivial, we have
\begin{align*}
{\bar \alpha} & = 1\\
{\bar \beta}^{a} & = 0\\
{\bar \Gamma}^{a}_{~bc} & = 0\\
{\bar g} & = 1\\
B^{a} & = 0
\end{align*}
Therefore the equation above becomes
\begin{align}\label{transformed invariant gamma-driver}
&{\dot \beta}^{a'} - \beta^{b'}\partial_{b'}\beta^{a'} - \sigma^{a}\frac{\partial x^{a'}}{\partial x^{a}}\left(\frac{\partial t}{\partial t'}\right)^{2} = \notag\\
& \frac{3}{4}\sqrt{g'}^{2/3}\left|\frac{\partial x'}{\partial x}\right|^{2/3}\left(\frac{\partial t}{\partial t'}\right)^{2}(\Gamma^{a'}_{~b'c'}g^{b'c'} - \frac{\partial x^{a'}}{\partial x^{a}}\frac{\partial^{2}x^{a}}{\partial x^{b'}\partial x^{c'}}g^{b'c'} )\notag\\
&\frac{1}{3}\Gamma^{c'}_{~b'c'}g^{a'b'}  - \frac{1}{3}\frac{\partial x^{c'}}{\partial x^{c}}\frac{\partial^{2}x^{c}}{\partial x^{c'}\partial x^{b'}}g^{a'b'})\notag\\
& - \eta\frac{\partial t}{\partial t'}\beta^{a'} + \eta \frac{\partial t}{\partial t'}\frac{\partial x^{a'}}{\partial x^{a}}\frac{\partial x^{a}}{\partial t'}
\end{align}
One can also easily tell that Eq.~\ref{transformed gamma-driver} and Eq.~\ref{transformed invariant gamma-driver} are identical except for a factor of $\frac{\partial x^{a}}{\partial x^{a'}}\left(\frac{\partial t'}{\partial t}\right)^{2}$.
\end{document}
